\chapter{State of the art}
\label{stato dell'arte}


Capitolo uno: 
esempio di bibliografia \cite{2015arXiv150200046K}, un esempio di figura \fref{figure:test inserimento} e un esempio di tabella \tref{table:esempio_tabella}.

\begin{figure}[H]
	\centering
	\includegraphics[width=8cm]{images/example.png}
	\caption{Test inserimento immagini}
	\label{figure:test inserimento}
\end{figure}

\begin{table}[!htbp]
	\centering
	\begin{tabular}{l|l}
		\toprule
		Colonna 1 & Colonna 2 \\
		\midrule
		Valore 1 & Valore 2 \\
		Valore 3 & Valore 4 \\
		Valore 5 & Valore 6 \\
		\bottomrule
	\end{tabular}
	\caption{Descrizione della tabella}
	\label{table:esempio_tabella}
\end{table}


% ------------------------------------------------------------------
\section{Neural Networks}\label{s:fundamentals}
% ------------------------------------------------------------------

Neural Networks are becoming part of our life. All the apps we use are progressively adopting neural networks for pattern recognition and prediction.
In this section I'll provide an introduction to how they work and why a Neural Network was perfect for my task: recognizing clothing in a real world image.

% ------------------------------------------------------------------
\subsection{Introduction}\label{s:fund-intro}
% ------------------------------------------------------------------
\begin{quotation}
Machine Learning is a subfield of computer science focusing on programming computers to learn from data without being explicitly programmed. Machine learning algorithms learn and recognize patterns in seen data (training data) and use these patterns to predict characteristics of unseen data. \cite{IntroductiontoMachineLearningAlanZhengSeptember2018}
\end{quotation}

They can learn in 3 main ways:
\begin{enumerate}
	\item Supervised Learning:
		The algorithm is fed with annotations created by humans. It uses those to learn and grasp correlations between patterns, and apply them to newly fed images.
	\item Unsupervised Learning:
		It works the opposite way, trying to assign objects to groups, thus creating "unlabeled labels". So it groups all items with certain common patterns together.
	\item Reinforcement Learning:
		Reinforced Learning is commonly used to make computers learn how to play games. In Reinforced Learning, a goal is specified and the agent is given rewards for following a succesful path,
\end{enumerate}
For our purpose, I seeked to find a labeled dataset, because to recognize clothes and apparel we needed labeled images to let the algorithm know which is which.


Here are some basic words we need to know:
\begin{itemize}
	\item Label:
		The thing we're trying to predict. Ex: If we're trying to predict what kind of animal is in a picture, the label would be the type of animal in the picture.
	\item Classification:
		Taking each instance and assigning it to a particular category. Ex: Determining if tumors are benign or malignant by looking at MRI scans.
	\item Regression:
		Instead of having discrete classes, like classification, the "class" to be predicted is made up of continuous numerical values. Ex: Predicting house prices based on square footage, number of rooms, etc.
	\item Clustering:
		For data without pre-labeled classes, clustering is the act of grouping similar data points together. A form of unsupervised learning. Ex: Clustering U.S. households for marketing data.
	\item Training Data:
		The initial set of data used to discover potentially predictive relationships. It's what your machine learning algorithm "trains" on and learns patterns from.
	\item Validation and Testing Data:
		Set of data used to assess the strength and utility of predictive relationship. Your machine learning algorithm does not see this data during training.
	\item Error:
		The difference between algorithm's prediction and ground-truth values.
	\item Ground Truth:
		Data that is known to be correct. A data-set's labels.
	\item Features:
		The attributes of the data that are used to make a prediction about the labels. Ex: In the house price example in the regression definition, the features would be the square footage, number of rooms, etc.
	\item Feature space:
		The n-dimensions in which the features live where n = the number of features. Typically, the larger the feature space, the more complex your algorithm will be.
	\item Model:
		The relationship between features and label. Training means "learning" this relationship based on examples.


	
\end{itemize}
Let's briefly explain some concepts before diving into Neural Networks:

\subsubsection{Towards NNs}\label{s:fund-intro-tonn}
\begin{enumerate}
	\item Overfitting:
		\begin{figure}[H]
			\centering
			\includegraphics[scale=0.4]{images/overfitting.png}
			\caption{Overfitting on a simple separation problem}
			\label{f:overfitting}
		\end{figure}
		Simply put, overfitting means overgeneralizing narrow concepts. In the image above we can clearly see that the semantic line we should follow is the black one, even though, for this \emph{particular} dataset, the green line would produce the best results.
		What the green separation is doing is not grasping the shape of the separation, but picking up the intrinsic noise of real problems.
		
		\begin{figure}[H]
			\centering
			\includegraphics[scale=1.2]{images/overfittinglines.png}
			\caption{Recognizing overfitting}
			\label{f:overfittinglines}
		\end{figure}
		It's easy to recognize overfitting when training. If the error rates start to diverge, it usually is because of it. Avoiding it requires having a broad dataset that is balanced on the amount of annotations of each type.
	\item Decision Trees:
		
		They narrow down objects into classes by using binary questions repeatedly.
	\item Random Forests:
		
		A number of Decision Trees grouped together to identify multiple features.
		\begin{figure}[H]
			\centering
			\includegraphics[scale=0.4]{images/featureimportance.jpg}
			\caption{Feature Importance using Random Trees}
			\label{f:featureimportance}
		\end{figure}
		They can be also used to measure the importance a particular feature. In fact, while usually the most important features in a tree are the top level ones, in random forests you have more than one tree, so a good way to find the importance of a feature is averaging the depth of the level where the feature is present in all the trees of the forest. In the image above, the features are the single pixels.
	\item Support Vector Machines:
	
		SVMs are widely used (\sref{s:nnevo-cnn}) to do supervised learning on both linear and non-linear classification.
		\begin{figure}[h!]
			\centering
			\includegraphics[scale=0.4]{images/svm.jpg}
			\caption{Linear Classification on Support Vector Machine}
			\label{fig:svm}
		\end{figure}
		\begin{figure}[h!]
			\centering
			\includegraphics[scale=0.18]{images/dimensions.jpg}
			\caption{Non-Linear Classification on Support Vector Machine}
			\label{fig:mapping}
		\end{figure}
	\item Intersection over Union (IoU) for object detection:
		
		IoU is an evaluating method used to measure the accuracy of an object detector on a particular dataset.
		
		\begin{figure}[htbp]
			\centering
			\begin{minipage}{0.45\textwidth}
				\centering
				\includegraphics[width=0.8\textwidth]{images/iou_stop_sign.jpg} % first
				%figure itself
			\end{minipage}\hfill
			\begin{minipage}{0.45\textwidth}
				\centering
				\includegraphics[width=0.8\textwidth]{images/iou_equation.png} %second
				%figure itself
			\end{minipage}
			\caption{Intersection over Union (IoU) as evaluation method for object detection accuracy in a dataset}
			\label{f:iou}
		\end{figure}
	
	
	
\end{enumerate}




\subsection{What problems do they solve}\label{s:ltask}


\begin{figure}[H]
	\begin{center}
		\includegraphics[width=0.35\textwidth]{neural_networks_basis/learning_problem}
		\caption{Learning problem for neural networks.}
		\label{LearningActivityDiagram}
	\end{center}
\end{figure}

Learning tasks for neural networks can be classified according to the source of information for them. \fref{LearningActivityDiagram}

There are basically two sources of information: data sets and mathematical models. 
Some classes of learning tasks which learn from data sets are \emph{function regression}, \emph{pattern recognition} or \emph{time series prediction}. 
Learning tasks in which learning is performed from mathematical models are \emph{optimal control} or \emph{optimal shape design}. 
Finally, in inverse problems the neural network learns from both data sets and mathematical models.



\subsubsection{Function regression}

Function regression is the most popular learning task for neural networks. 
It is also called modelling. The function regression problem can be regarded as the problem of
approximating a function from a data set consisting of input-target instances
\cite{Haykin1994}. The targets are a specification
of what the response to the inputs should be \cite{Bishop1995}. 
While input variables might be quantitative or qualitative, in function regression target variables are quantitative. 

Performance measures for function regression are based on a sum of errors between the outputs from the neural network and the targets in the training data. 
As the training data is usually deficient, a regularization term might be required in order to solve the problem correctly.

An example is to design an instrument that can determine serum cholesterol levels from
measurements of spectral content of a blood sample. There are a number of 
patients for which there are measurements of several wavelengths of the spectrum.
For the same patients there are also measurements of several
cholesterol levels, based on serum separation \cite{Demuth2009}.

Function regression will be used in our problem internally, specifically relative to Bounding-box regression.

% TODO insert bounding-box regression section reference

\subsubsection{Pattern recognition}\label{s:ltask-patt}

The learning task of pattern recognition gives raise to artificial intelligence. That problem can be stated as the process whereby a received pattern, characterized by a distinct
set of features, is assigned to one of a prescribed number of
classes \cite{Haykin1994}. Pattern recognition is also known as classification. Here the neural network learns from knowledge represented by a training data set consisting of input-target instances. The inputs include a set of features which characterize a pattern, and they can be quantitative or qualitative. The targets specify the class that each pattern belongs to and therefore are qualitative \cite{Bishop1995}.

Classification problems can be, in fact, formulated as being modelling problems. 
The learning task of pattern recognition is generally more difficult to solve than that of function regression. 
This means that a good knowledge of the state of the technique is recommended for success. 

A typical example is to disinguish hand-written versions of characters. 
Images of the characters might be captured and fed to a computer. 
We aim to find an algorithm which can distinguish the characters as reliably as possible  \cite{Bishop1995}. 

I'll talk more speficically about pattern recognition problems in \sref{s:patt}, after explaining the basic features of Neural Networks [\sref{s:perc}] and CNNs [\sref{s:cnn-structure}].

\subsubsection{Optimal control}

Optimal control is playing an increasingly important role in the
design of modern engineering systems. The aim
here is the optimization of a physical
process. More specifically, the objective of these problems is to
determine the control signals that will cause a process to satisfy
the physical constraints and at the same time minimize or maximize
some performance criterion \cite{Kirk1970} \cite{BalsaCanto2001}. 

The knowledge in optimal control problems is not represented in the form of a data set, it is given by a mathematical model. 
These objective functionals are often defined by integrals, ordinary differential equations or partial differential equations. 
This way, in order to evaluate them, we might need to apply Simpon methods, Runge-Kutta methods or other finite element methods. 
Optimal control problems often include constraints. 

As a simple example, consider the problem of a rocket launching a
satellite into an orbit around the earth. An associated optimal
control problem is to choose the controls (the thrust attitude angle
and the rate of emission of the exhaust gases) so that the rocket
takes the satellite into its prescribed orbit with minimum
expenditure of fuel or in minimum time.

\subsubsection{Optimal shape design}

Optimal shape design is a very interesting field for industrial
applications. The goal in these problems
is to computerize the development process of some tool, and
therefore shorten the time it takes to create or to improve the
existing one. Being more precise, in an optimal shape design process
one wishes to optimize some performance criterium involving the
solution of a mathematical model with respect to its domain of
definition \cite{Bucur2005}. 

As in the previous case, the neural network here learns from a mathematical model. 
Evaluation of the performance functional here might also need the integration of functions, ordinary differential equations or partial differential equations. 
Optimal shape design problems defined by partial differential equations are challenging applications. 

One example is the design of airfoils,
which proceeds from a knowledge of computational fluid dynamics
\cite{Eyi1994} \cite{Mohammadi2004}. The performance goal here might
vary, but increasing lift and reducing drag are among the most common. Other objectives as weight reduction, stress reinforcement and
even noise reduction can be obtained. On the other hand, the airfoil
may be required to achieve this performance with constraints on
thickness, pitching moment, etc.

\subsubsection{Inverse problems}

Inverse problems can be described as being opposed to direct
problems. In a direct problem the cause is given, and the effect is
determined. In an inverse problem the effect is given, and the cause
is estimated \cite{Kirsch1996} \cite{Sabatier2000} \cite{Ramm2005}.
There are two main types of inverse problems: input estimation, in
which the system properties and output are known and the input is to
be estimated; and properties estimation, in which the the system
input and output are known and the properties are to be estimated.
Inverse problems can be found in many areas of science and
engineering. 

This type of problems is of great interest from both a theoretical and practical perspectives. 
From a theoretical point of view, the neural network here needs both mathematical models and data sets. 
The aim is usually formulated as to find properties or inputs which make a mathematical model to comply with the data set. 
From a practical point of view, most numerical software must be tuned up before being on production. 
That means that the particular properties of a system must be properly estimated in order to simulate it well.

A typical inverse problem in geophysics is to find the 
subsurface inhomogeneities from collected scattered fields caused by
acoustic waves sent at the surface and a mathematical model of soil mechanics.  

\subsubsection{Tasks overview}

The knowledge for a neural network can be represented in the form of data sets or mathematical models. 
The neural network learns from data sets in function regression and pattern recognition; 
it learns from mathematical models in optimal control and optimal shape design; 
and it learns from both mathematical models and data sets in inverse problems. 

\fref{LearningTasksFigure} shows the learning tasks for neural networks described in this section. 
As we can see, they are capable of dealing with a great range of applications. 
Modelling and classification are the most traditional; 
optimal control, optimal shape design and inverse problems can also be very useful.

Our learning task, that \maskrcnn solves really well, is Pattern Recognition (\sref{s:ltask-patt}).

\begin{figure}[H]
	\begin{center}
		\includegraphics[width=1.0\textwidth]{neural_networks_basis/learning_tasks}
		\caption{Learning tasks for neural networks.}\label{LearningTasksFigure}
	\end{center}
\end{figure}



\subsection{The Perceptron}\label{s:perc}
\begin{figure}[H]
	\centering
	\includegraphics[scale=0.5]{images/2MVdW}
	\caption{The Perceptron}
	\label{f:perceptron}
\end{figure}
\subsubsection{Definition}

A perceptron is the fundamental unit of a Neural Network (which is even called a Multi-Layer Perceptron for this reason). Refer to the diagram above. Perceptrons contain two or more inputs, a weight for each input, a bias, an activation function (the step function) and an output.
For the perceptron above with $2$ inputs, the intermediate value $f(x)$ is as follows
\[f(x) = w_1x_1 + w_2x_2 + b\]
The final output $y$ is just the step function:
\[
y =
\begin{cases}
0 & \text{if $f(x) < 0$} \\
1 & \text{if $f(x) > 0$}
\end{cases}
\]
\subsubsection{Visualization}

The purpose of a perceptron is to classify data. Consider the function AND.

\begin{table}[H]
	\centering
	\begin{tabular}{ |c|c|c| } 
		\hline
		x1 & x2 & out \\
		\hline
		0 & 0 & 0 \\ 
		0 & 1 & 0 \\ 
		1 & 0 & 0 \\ 
		1 & 1 & 1 \\ 
		\hline
	\end{tabular}
	\caption{Function AND}
	\label{t:and}
\end{table}
Let's graph this data.
\begin{center}
	\begin{tikzpicture}
	\begin{axis}[
	axis lines=middle,
	xmin=-1, xmax=2,
	ymin=-1, ymax=2,
	xtick=\empty, ytick=\empty
	]
	\addplot [only marks] table {
		0 0
		0 1
		1 0
	};
	\addplot [only marks, mark=o] table {
		1 1
	};
	\addplot [domain=-10:10, samples=2, dashed] {-1*x+1.5};
	\end{axis}
	\end{tikzpicture}
\end{center}

The line $y = -x + 1.5$ splits this data the best.
Let's rearrange this to get $x + y - 1.5 = 0$. 
Going back to the perceptron formula
\[f(x) = w_1x_1 + w_2x_2 + b\]
we can see that for the optimal perceptron,  $w_1$ and $w_2$ are the coefficients of $x$ and $y$, and $b=-1.5$. If $f(x) > 0$, then $x + y - 1.5>0$. We can see through this example that perceptrons are nothing more than linear functions. Above a line, perceptrons classify data points $1$, below the line, they are $0$.

\subsubsection{Learning}

How do perceptrons "learn" the best possible linear function to split the data? Perceptrons adjust the weights and bias to iteratively approach a solution.

Let's consider this data:
\begin{center}
	\begin{tikzpicture}
	\begin{axis}[
	axis lines=middle,
	xmin=-3, xmax=3,
	ymin=-3, ymax=3,
	xtick=1, ytick=1,
	]
	\addplot [only marks] table {
		0 0
		-1 2 
		1 -2
		-1 -2
		-2 0
	};
	\addplot [only marks, mark=o] table {
		1 1
		0 2
		2 2
		1 3
		1 0
	};
	\addplot [domain=-10:10, samples=2, dashed] {-1*x+1.5};
	\end{axis}
	\end{tikzpicture}
\end{center}

The perceptron that represents the dashed line $y+x-1.5=0$ has two inputs, $x_1, x_2,$ with corresponding weights $w_1=1, w_2=1$, and bias $b = -1.5$. Let $y$ represent the output of this perceptron. In the data above, the point $(1, 0)$ is the only misclassified point. The perceptron outputs 0 because it is below the line, but it should output a 1.

For some data point (input) $i$ with coordinates $(i_1, i_2)$, the perceptron adjusts its weights and bias according to this formula:
\[w_1 = w_1 + \alpha(d-y)(i_1)\]
\[w_2 = w_2 + \alpha(d-y)(i_2)\]
\[b = b + \alpha(d-y)\]
Where $d$ is the desired output, and $\alpha$ is the learning rate, a constant usually between $0$ and $1$. Notice that the equation degenerates to $w = w$ and $b=b$ when the desired output equals the perceptron output. In other words, the perceptron only learns from misclassified points.

In the case of the above data, the perceptron only learns from the point $(1, 0)$. Let's set $\alpha=0.2$ and compute the learning steps:
\[w_1 = 1 + 0.2(1-0)(1) = 1.2\]
\[w_2 = 1 + 0.2(1-0)(0) = 1\]
\[b = -1.5 + 0.2(1-0) = -1.3\]

After 1 iteration, the perceptron now represents the function $y+1.2x-1.3 = 0$, which is shown below:
\begin{center}
	\begin{tikzpicture}
	\begin{axis}[
	axis lines=middle,
	xmin=-3, xmax=3,
	ymin=-3, ymax=3,
	xtick=1, ytick=1,
	]
	\addplot [only marks] table {
		0 0
		-1 2 
		1 -2
		-1 -2
		-2 0
	};
	\addplot [only marks, mark=o] table {
		1 1
		0 2
		2 2
		1 3
		1 0
	};
	\addplot [domain=-10:10, samples=2, dashed] {-1.2*x+1.3};
	\end{axis}
	\end{tikzpicture}
\end{center}

The next iteration follows:
\[w_1 = 1.2 + 0.2(1-0)(1) = 1.4\]
\[w_2 = 1 + 0.2(1-0)(0) = 1\]
\[b = -1.3 + 0.2(1-0) = -1.1\]

\begin{center}
	\begin{tikzpicture}
	\begin{axis}[
	axis lines=middle,
	xmin=-3, xmax=3,
	ymin=-3, ymax=3,
	xtick=1, ytick=1,
	]
	\addplot [only marks] table {
		0 0
		-1 2 
		1 -2
		-1 -2
		-2 0
	};
	\addplot [only marks, mark=o] table {
		1 1
		0 2
		2 2
		1 3
		1 0
	};
	\addplot [domain=-10:10, samples=2, dashed] {-1.4*x+1.1};
	\end{axis}
	\end{tikzpicture}
\end{center}

All the points are now correctly classified. The perceptron has learned! Notice how it has not learned the best possible line, only the first one that zeroes the difference between expected and actual output.
\subsubsection{Non-Linearly Separable Data}
Consider the function XOR:

\begin{table}[H]
	\centering
	\begin{tabular}{ |c|c|c| } 
		\hline
		x1 & x2 & out \\
		\hline
		0 & 0 & 1 \\ 
		0 & 1 & 0 \\ 
		1 & 0 & 0 \\ 
		1 & 1 & 1 \\ 
		\hline
	\end{tabular}
	\caption{Function XOR}
	\label{t:xor}
\end{table}
Let's graph this data.
\begin{center}
	\begin{tikzpicture}
	\begin{axis}[
	axis lines=middle,
	xmin=-1, xmax=2,
	ymin=-1, ymax=2,
	xtick=\empty, ytick=\empty
	]
	\addplot [only marks] table {
		0 1
		1 0
	};
	\addplot [only marks, mark=o] table {
		0 0
		1 1
	};
	\addplot [domain=-10:10, samples=2, dashed] {-1*x+1.5};
	\addplot [domain=-10:10, samples=2, dashed] {-1*x+0.5};
	\end{axis}
	\end{tikzpicture}
\end{center}

We need two lines to separate this data! A perceptron will never reach the optimal solution. However, multiple perceptrons can learn multiple lines, which can be used to classify non-linearly separable data.

\subsubsection{Multi-Layer Perceptron}
A neural network (NN) or Multi-Layer Perceptron (MLP) is a bunch of these perceptrons glued together, and can be used to approximate multi-dimensional non-linearly separable data.
Let's again consider XOR. How do we arrange perceptrons to represent the two functions?

Clearly, we need two perceptrons, one for each function. The output of these two perceptrons can be used as inputs to a third perceptron, which will give us our output. Refer to the diagram below.

\begin{figure}[H]
	\centering
	\includegraphics[scale=0.3]{images/Frame}
	\caption{Multi-Layer Perceptron}
	\label{f:Frame}
\end{figure}

Let perceptron 1 model $y + x - 1.5 = 0$ (the upper line), and perceptron 2 model $y + x - 0.5 = 0$ (the lower line). Because the weights are the coefficients of these functions, $w_1 = 1, w_2 = 1, w_3 = 1, w_4 = 1$ and the biases $b_1 = -1.5$ and $b_2 = -0.5$.

The output of Perceptron 1 will be a 1 for points above the upper line, and a 0 for the points below the upper line. The output of Perceptron 2 will be a 1 for points above the lower line, and a 0 for points below the lower line. In between the lines, these cancel! However, in order to create a threshold to separate the points between the lines from the points outside, we would like the outputs for points between the lines to be additive.

In other words, we would like the inputs of Perceptron 3 to both be 1 between the lines, and have a maximum of a single 1 for points outside the lines. Thus, we let $w_6 = 1$ and $w_5 = -1$. This gives us an output of $2$ for points between the lines, and a maximum output of 1 for points outside the lines. Thus, we can set the bias for Perceptron 3: $b_3 = -1.5$.


% ------------------------------------------------------------------
\subsection{How they work}\label{s:cnn-structure}
% ------------------------------------------------------------------

A \emph{Neural Network} (NN) is a function $g$ mapping data $\bx$, for example an image, to an output vector $\by$, for example an image label. The function $g=f_L \circ \dots \circ f_1$ is the composition of a sequence of simpler functions $f_l$, which are called \emph{computational blocks} [\sref{s:blocks}] or \emph{layers}. Let $\bx_1,\bx_2,\dots,\bx_L$ be the outputs of each layer in the network, and let $\bx_0=\bx$ denote the network input. Each intermediate output $\bx_l = f_l(\bx_{l-1};\bw_l)$ is computed from the previous output $\bx_{l-1}$  by applying the function $f_l$ with parameters $\bw_l$. 

\subsubsection{Convolutional Neural Network}

In a \emph{Convolutional Neural Network} (CNN), the data has a spatial structure: each $\bx_l\in\mathbb{R}^{H_l \times W_l \times C_l}$ is a 3D array or \emph{tensor} where the first two dimensions $H_l$ (height) and $W_l$ (width) are interpreted as spatial dimensions. The third dimension $C_l$ is instead interpreted as the \emph{number of feature channels}. Hence, the tensor $\bx_l$ represents a $H_l \times W_l$ field of $C_l$-dimensional feature vectors, one for each spatial location. A fourth dimension $N_l$ in the tensor spans multiple data samples packed in a single \emph{batch} for efficiency parallel processing. The number of data samples $N_l$ in a batch is called the batch \emph{cardinality}. The network is called \emph{convolutional} because the functions $f_l$ are local and translation invariant operators (i.e.\ non-linear filters) like linear convolution.

It is also possible to conceive CNNs with more than two spatial dimensions, where the additional dimensions may represent volume or time. In fact, there are little \emph{a-priori} restrictions on the format of data in neural networks in general. Many useful NNs contain a mixture of convolutional layers together with layer that process other data types such as text strings, or perform other operations that do not strictly conform  to the CNN assumptions.

There are a variety of layers such as (convolution),  (convolution transpose or deconvolution),  (max and average pooling),  (ReLU activation),  (sigmoid activation),  (softmax operator),  (classification log-loss), (batch normalization), (spatial normalization), (local response normalization -- LRN). 

NNs are often used as classifiers or regressors. In the example of \sref{f:demo}, the output $\hat \by = f(\bx)$ is a vector of probabilities, one for each of a 1,000 possible image labels (dog, cat, trilobite, ...).  If $\by$ is the true label of image $\bx$, we can measure the CNN performance by a loss function $\ell_\by(\hat \by)  \in \mathbb{R}$ which assigns a penalty to classification errors. The CNN parameters can then be tuned or \emph{learned} to minimize this loss averaged over a large dataset of labelled example images.

% ------------------------------------------------------------------
\paragraph{Optimizers}\label{s:cnn-optimizers}
% ------------------------------------------------------------------

Learning generally uses a variant of \emph{stochastic gradient descent} (SGD). While this is an efficient method (for this type of problems), networks may contain several million parameters and need to be trained on millions of images. SGD also requires to compute the CNN derivatives, as explained in the next section. Our \maskrcnn uses Adam as optimizer, not SGD, although SGD seems to have better segmentation performance, albeit a bit slower to learn \cite{wilson2017marginal}.
\begin{quotation}“We observe that the solutions found by adaptive methods generalize worse (often significantly worse) than SGD, even when these solutions have better training performance. These results suggest that practitioners should reconsider the use of adaptive methods to train neural networks."
\end{quotation}

\begin{figure}[H]
	\centering
	\includegraphics[width=0.5\columnwidth]{figures/pepper}
	\caption{An example of one of  \matlab stock images using a large CNN pre-trained on ImageNet.}
	\label{f:demo}
\end{figure}

% ------------------------------------------------------------------
\subsubsection{Network structures}\label{s:cnn-topology}
% ------------------------------------------------------------------

In the simplest case, layers in a NN are arranged in a sequence; however, more complex interconnections are possible as well, and in fact very useful in many cases. This section discusses such configurations and introduces a graphical notation to visualize them.

% ------------------------------------------------------------------
\paragraph{Sequences}\label{s:cnn-simple}
% ------------------------------------------------------------------

Start by considering a computational block $f$ in the network. This can be represented schematically as a box receiving data $\bx$ and parameters $\bw$ as inputs and producing data $\by$ as output:
\begin{center}
	\begin{tikzpicture}[auto, node distance=2cm]
	\node (x) [data] {$\bx$};
	\node (f) [block,right of=x]{$f$};
	\node (y) [data, right of=f] {$\by$};
	\node (w) [data, below of=f] {$\bw$};
	\draw [to] (x.east) -- (f.west) {};
	\draw [to] (f.east) -- (y.west) {};
	\draw [to] (w.north) -- (f.south) {};
	\end{tikzpicture}
\end{center}
As seen above, in the simplest case blocks are chained in a sequence $f_1 \rightarrow f_2\rightarrow\dots\rightarrow f_L$ yielding the structure:
\begin{center}
	\begin{tikzpicture}[auto, node distance=2cm]
	\node (x0)  [data] {$\bx_0$};
	\node (f1) [block,right of=x0]{$f_1$};
	\node (f2) [block,right of=f1,node distance=3cm]{$f_2$};
	\node (dots) [right of=f2]{...};
	\node (fL) [block,right of=dots]{$f_L$};
	\node (xL)  [data, right of=fL] {$\bx_L$};
	\node (w1) [data, below of=f1] {$\bw_1$};
	\node (w2) [data, below of=f2] {$\bw_2$};
	\node (wL) [data, below of=fL] {$\bw_L$};
	\draw [to] (x0.east) -- (f1.west) {};
	\draw [to] (f1.east) -- node {$\bx_1$} (f2.west);
	\draw [to] (f2.east) -- node {$\bx_2$} (dots.west) {};
	\draw [to] (dots.east) -- node {$\bx_{L-1}$} (fL.west) {};
	\draw [to] (fL.east) -- (xL.west) {};
	\draw [to] (w1.north) -- (f1.south) {};
	\draw [to] (w2.north) -- (f2.south) {};
	\draw [to] (wL.north) -- (fL.south) {};
	\end{tikzpicture}
\end{center}
Given an input $\bx_0$, evaluating the network is a simple matter of evaluating all the blocks from left to right, which defines a composite function $\bx_L = f(\bx_0;\bw_1,\dots,\bw_L)$. 

% ------------------------------------------------------------------
\paragraph{Directed acyclic graphs}\label{s:cnn-dag}
% ------------------------------------------------------------------

\begin{figure}[t]
	\begin{center}
		\begin{tikzpicture}[auto, node distance=0.4cm]
		\matrix (m) [matrix of math nodes, 
		column sep=1.2cm,
		row sep=0.4cm]
		{
			& \node (f1) [block]{f_1}; 
			& \node (x1) [datac]{\bx_1};
			\\
			\node (x0) [datac]{\bx_0};
			&
			&
			& \node (f3) [block]{f_3};
			& \node (x3) [datac]{\bx_3};
			\\
			& \node (f2) [block]{f_2}; 
			& \node (x2) [datac]{\bx_2};
			& &
			& \node (f5) [block]{f_5}; 
			& \node (x7) [datac]{\bx_7}; 
			\\
			& 
			& \node(x5) [datac]{\bx_5};
			\\
			\node (x4) [datac]{\bx_4};
			& \node (f4) [block]{f_4};
			\\
			& 
			& \node(x6) [datac]{\bx_6};
			\\
		};
		\draw[to] (x0) -- (f1);
		\draw[to] (f1) -- (x1);
		\draw[to] (x1) -- (f3);
		\draw[to] (x0) -- (f2);
		\draw[to] (f2) -- (x2);
		\draw[to] (x2) -- (f3);
		\draw[to] (f3) -- (x3);
		\draw[to] (x3) -- (f5);
		\draw[to] (f5) -- (x7);
		\draw[to] (x4) -- (f4);
		\draw[to] (f4) -- (x5);
		\draw[to] (f4) -- (x6);
		\draw[to] (x5) -- (f5);
		\node(w1) [par,below=of f1]{$\bw_1$}; \draw[to] (w1) -- (f1);
		\node(w2) [par,below=of f2]{$\bw_2$}; \draw[to] (w2) -- (f2);
		%\node(w3) [par,below=of f3]{$\bw_3$}; \draw[to] (w3) -- (f3);
		\node(w4) [par,below=of f4]{$\bw_4$}; \draw[to] (w4) -- (f4);
		\draw[to] (w4) to [bend right] (f3);
		\node(w5) [par,below=of f5]{$\bw_5$}; \draw[to] (w5) -- (f5);
		\end{tikzpicture}
	\end{center}
	\vspace{-1em}
	\caption{\textbf{Example DAG.}}\label{f:dag}
\end{figure}

One is not limited to chaining layers one after another. In fact, the only requirement for evaluating a NN is that, when a layer has to be evaluated, all its input have been evaluated prior to it. This is possible exactly when the interconnections between layers form a \emph{directed acyclic graph}, or DAG for short.

In order to visualize DAGs, it is useful to introduce additional nodes for the network variables, as in the  example of \fref{f:dag}. Here boxes denote functions and circles denote variables (parameters are treated as a special kind of variables). In the example, $\bx_0$ and $\bx_4$ are the inputs of the CNN and $\bx_6$ and $\bx_7$ the outputs. Functions can take any number of inputs (e.g. $f_3$ and $f_5$ take two) and have any number of outputs (e.g. $f_4$ has two). There are a few noteworthy properties of this graph:

\begin{enumerate}
	\item The graph is bipartite, in the sense that arrows always go from boxes to circles and from circles to boxes. 
	\item Functions can have any number of inputs or outputs; variables and parameters can have an arbitrary number of outputs (a parameter with more of one output is \emph{shared} between different layers); variables have at most one input and parameters none. 
	\item Variables with no incoming arrows and parameters are not computed by the network, but must be set prior to evaluation, i.e.\ they are \emph{inputs}. Any variable (or even parameter) may be used as output, although these are usually the variables with no outgoing arrows.
	\item Since the graph is acyclic, the CNN can be evaluated by sorting the functions and computing them one after another (in the example, evaluating the functions in the order $f_1,f_2,f_3,f_4,f_5$ would work).
\end{enumerate}

% ------------------------------------------------------------------
\subsubsection{Computing derivatives with backpropagation}\label{s:back}
% ------------------------------------------------------------------

Learning a NN requires computing the derivative of the loss with respect to the network parameters. Derivatives are computed using an algorithm called \emph{backpropagation}, which is a memory-efficient implementation of the chain rule for derivatives. First, we discuss the derivatives of a single layer, and then of a whole network.

\paragraph{Derivatives of tensor functions}

In a CNN, a layer is a function $\by = f(\bx)$ where both input $\bx \in \mathbb{R}^{H\times W \times C}$ and output $\by \in \mathbb{R}^{H'\times W' \times C'}$ are tensors. The derivative of the function $f$ contains the derivative of each output component $y_{i'j'k'}$ with respect to each input component $x_{ijk}$, for a total of $H'\times W'\times C'\times H\times W\times C$ elements naturally arranged in a 6D tensor. Instead of expressing derivatives as tensors, it is often useful  to switch to a matrix notation by \emph{stacking} the input and output tensors into vectors. This is done by the $\vv$ operator, which visits each element of a tensor in lexicographical order and produces a vector:
\[
\vv \bx
=
\begin{bmatrix}
x_{111} \\
x_{211} \\
\vdots
\\
x_{H11} \\
x_{121} \\
\vdots \\
x_{HWC}  	
\end{bmatrix}.
\]
By stacking both input and output, each layer $f$ can be seen reinterpreted as vector function $\vv f$, whose derivative is the conventional Jacobian matrix:
\[
\renewcommand*{\arraystretch}{1.5}
\frac{d \vv f}{d(\vv \bx)^\top}
=
\begin{bmatrix}
\frac{\partial y_{111}}{\partial x_{111}} & 
\frac{\partial y_{111}}{\partial x_{211}} &
\dots &
\frac{\partial y_{111}}{\partial x_{H11}} &
\frac{\partial y_{111}}{\partial x_{121}} &
\dots &
\frac{\partial y_{111}}{\partial x_{HWC}} \\
\frac{\partial y_{211}}{\partial x_{111}} & 
\frac{\partial y_{211}}{\partial x_{211}} &
\dots &
\frac{\partial y_{211}}{\partial x_{H11}} &
\frac{\partial y_{211}}{\partial x_{121}} &
\dots &
\frac{\partial y_{211}}{\partial x_{HWC}} \\
\vdots & \vdots & \dots & \vdots & \vdots & \dots & \vdots \\
\frac{\partial y_{H'11}}{\partial x_{111}} & 
\frac{\partial y_{H'11}}{\partial x_{211}} &
\dots &
\frac{\partial y_{H'11}}{\partial x_{H11}} &
\frac{\partial y_{H'11}}{\partial x_{121}} &
\dots &
\frac{\partial y_{H'11}}{\partial x_{HWC}} \\
\frac{\partial y_{121}}{\partial x_{111}} & 
\frac{\partial y_{121}}{\partial x_{211}} &
\dots &
\frac{\partial y_{121}}{\partial x_{H11}} &
\frac{\partial y_{121}}{\partial x_{121}} &
\dots &
\frac{\partial y_{121}}{\partial x_{HWC}} \\
\vdots & \vdots & \dots & \vdots & \vdots & \dots & \vdots \\
\frac{\partial y_{H'W'C'}}{\partial x_{111}} & 
\frac{\partial y_{H'W'C'}}{\partial x_{211}} &
\dots &
\frac{\partial y_{H'W'C'}}{\partial x_{H11}} &
\frac{\partial y_{H'W'C'}}{\partial x_{121}} &
\dots &
\frac{\partial y_{H'W'C'}}{\partial x_{HWC}}
\end{bmatrix}.
\]
This notation for the derivatives of tensor functions is taken from~\cite{kinghorn96integrals} and is used throughout this section.

While it is easy to express the derivatives of tensor functions as matrices, these matrices are in general extremely large. Even for moderate data sizes (e.g. $H=H'=W=W'=32$ and $C=C'=128$), there are $H'W'C'HWC \approx 17 \times 10^9$ elements in the Jacobian. Storing that requires 68 GB of space in single precision. The purpose of the backpropagation algorithm is to compute the derivatives required for learning without incurring this huge memory cost.

\paragraph{Derivatives of function compositions}

In order to understand backpropagation, consider first a simple CNN terminating in a loss function $f_L = \ell_\by$:
\begin{center}
	\begin{tikzpicture}[auto, node distance=2cm]
	\node (x0)  [data] {$\bx_0$};
	\node (f1) [block,right of=x0]{$f_1$};
	\node (f2) [block,right of=f1,node distance=3cm]{$f_2$};
	\node (dots) [right of=f2]{...};
	\node (fL) [block,right of=dots]{$f_L$};
	\node (w1) [data, below of=f1] {$\bw_1$};
	\node (w2) [data, below of=f2] {$\bw_2$};
	\node (wL) [data, below of=fL] {$\bw_L$};
	\node (xL) [data, right of=fL] {$x_L\in\real$};
	\draw [to] (x0.east) -- (f1.west) {};
	\draw [to] (f1.east) -- node {$\bx_1$} (f2.west);
	\draw [to] (f2.east) -- node {$\bx_2$} (dots.west) {};
	\draw [to] (dots.east) -- node {$\bx_{L-1}$} (fL.west) {};
	\draw [to] (fL.east) -- (xL.west) {};
	\draw [to] (w1.north) -- (f1.south) {};
	\draw [to] (w2.north) -- (f2.south) {};
	\draw [to] (wL.north) -- (fL.south) {};
	\end{tikzpicture}
\end{center}
The goal is to compute the gradient of the loss value $x_L$ (output) with respect to each network parameter $\bw_l$:
\[
\frac{df}{d(\vv \bw_l)^\top} = 
\frac{d}{d(\vv \bw_l)^\top}
\left[f_L(\cdot;\bw_L) \circ ... \circ 
f_2(\cdot;\bw_2) \circ f_1(\bx_0;\bw_1)\right].
\]
By applying the chain rule and by using the matrix notation introduced above, the derivative can be written as
\begin{equation}\label{e:chain-rule}
\frac{df}{d(\vv \bw_l)^\top} 
= 
\frac{d\vv f_L(\bx_{L-1};\bw_{L})}{d(\vv\bx_{L-1})^\top}
\times
\dots
\times
\frac{d\vv f_{l+1}(\bx_{l};\bw_{l+1})}{d(\vv\bx_{l})^\top}
\times
\frac{d\vv f_l(\bx_{l-1};\bw_{l})}{d(\vv\bw_l^\top)}
\end{equation}
where the derivatives are computed at the working point determined by the input $\bx_0$ and the current value of the parameters. 

Note that, since the network output $x_L$ is a \emph{scalar} quantity, the target derivative $df/d(\vv \bw_l)^\top$ has the same number of elements of the parameter vector $\bw_l$, which is moderate. However, the intermediate Jacobian factors have, as seen above, an unmanageable size. In order to avoid computing these factor explicitly, we can proceed as follows.

Start by multiplying the output of the last layer by a tensor $p_L=1$ (note that this tensor is a scalar just like the variable $x_L$):
\begin{align*}
p_L \times \frac{df}{d(\vv \bw_l)^\top} 
&= 
\underbrace{p_L \times \frac{d\vv f_L(\bx_{L-1};\bw_{L})}{d(\vv\bx_{L-1})^\top}}_{(\vv \bp_{L-1})^\top}
\times
\dots
\times
\frac{d\vv f_{l+1}(\bx_{l};\bw_{l+1})}{d(\vv\bx_{l})^\top}
\times
\frac{d\vv f_l(\bx_{l-1};\bw_{l})}{d(\vv\bw_l^\top)}
\\
&=
(\vv \bp_{L-1})^\top
\times
\dots
\times
\frac{d\vv f_{l+1}(\bx_{l};\bw_{l+1})}{d(\vv\bx_{l})^\top}
\times
\frac{d\vv f_l(\bx_{l-1};\bw_{l})}{d(\vv\bw_l^\top)}
\end{align*}
In the second line the last two factors to the left have been multiplied obtaining a new tensor $\bp_{L-1}$ that has the same size as the variable $\bx_{L-1}$. The factor $\bp_{L-1}$ can therefore be explicitly stored. The construction is then repeated by multiplying pairs of factors from left to right, obtaining a sequence of tensors $\bp_{L-2},\dots,\bp_{l}$ until the desired derivative is obtained. Note that, in doing so, no large tensor is ever stored in memory. This process is known as \emph{backpropagation}.

In general, tensor $\bp_{l}$ is obtained from $\bp_{l+1}$ as the product:
\[
(\vv \bp_{l})^\top = (\vv \bp_{l+1})^\top \times \frac{d\vv f_{l+1}(\bx_{l};\bw_{l+1})}{d(\vv\bx_{l})^\top}.
\]
The key to implement backpropagation is to be able to compute these products without explicitly computing and storing in memory the second factor, which is a large Jacobian matrix. Since computing the derivative is a linear operation, this product can be interpreted as the \emph{derivative of the layer projected along direction $\bp_{l+1}$}: 
\begin{equation}\label{e:projected}
\bp_{l} = 
\frac{d \langle \bp_{l+1}, f(\bx_l;\bw_l) \rangle}
{d \bx_{l}}.
\end{equation}
Here $\langle \cdot,\cdot \rangle$ denotes the inner product between tensors, which results in a scalar quantity. Hence the derivative \eqref{e:projected} needs not to use the $\vv$ notation, and yields a tensor $\bp_l$ that has the same size as $\bx_l$ as expected.

In order to implement backpropagation, the CNN for each layer $f$ can have:
\begin{itemize}
	\item A \textbf{forward mode}, computing the output $\by = f(\bx;\bw)$ of the layer given its input $\bx$ and parameters $\bw$.
	\item A \textbf{backward mode}, computing the projected derivatives
	\[
	\frac{d \langle \bp, f(\bx;\bw) \rangle}
	{d \bx}
	\quad\text{and}\quad
	\frac{d \langle \bp, f(\bx;\bw) \rangle}
	{d \bw},
	\]
	given, in addition to the input $\bx$ and parameters $\bw$, a tensor $\bp$ that the same size as $\by$.
\end{itemize}
%This is best illustrated with an example. Consider a layer $f$ such as the convolution operator. In the ``forward'' mode, one calls the function as $y = vl_nnconv(x,w,[])$ to apply the filters $w$ to the input $x$ and obtain the output $y$. In the ``backward mode'', one calls $[dx, dw] = vl_nnconv(x,w,[],p)$.  As explained above, $dx$, $dw$, and $p$ have the same size as $x$, $w$, and $y$, respectively. The computation of large Jacobian is encapsulated in the function call and never carried out explicitly. 

\paragraph{Backpropagation networks}\label{s:bpnets}

In this section, we provide a schematic interpretation of backpropagation and show how it can be implemented by ``reversing'' the NN computational graph.

The projected derivative of eq.~\eqref{e:projected} can be seen as the derivative of the following mini-network:
\begin{center}
	\begin{tikzpicture}[auto, node distance=2cm]
	\node (x) [data] {$\bx$};
	\node (f) [block,right of=x ] {$f$};
	\node (dot)[block,right of=f ] {$\langle \cdot, \cdot \rangle$};
	\node (z) [data, right of=dot] {$z \in \mathbb{R}$};
	\node (w) [data, below of=f ] {$\bw$};
	\node (p) [data, below of=dot] {$\bp$};
	\draw [to] (x.east) -- (f.west) {};
	\draw [to] (f.east) -- node {$\by$}  (dot.west) {};
	\draw [to] (w.north) -- (f.south) {};
	\draw [to] (dot.east) -- (z.west) {};
	\draw [to] (p.north) -- (dot.south) {};
	\end{tikzpicture}
\end{center}
In the context of back-propagation, it can be useful to think of the projection $\bp$ as the ``linearization'' of the rest of the network from variable $\by$ down to the loss. The projected derivative can also be though of as a new layer $(d\bx, d\bw) = df(\bx,\bw,\bp)$ that, by computing the derivative of the mini-network, operates in the reverse direction:
\begin{center}
	\begin{tikzpicture}[auto, node distance=2cm]
	\node (df) [block,right of=x] {$df$};
	\node (dx) [data,left of=df] {$d\bx$};
	\node (dw) [data,below of=df] {$d\bw$};
	\node (w) [data,above of=df,xshift=0.6em] {$\bw$};
	\node (x) [data,above of=df,xshift=-0.6em] {$\bx$};
	\node (p) [data,right of=df] {$\bp$};
	\draw [to] (df.west) -- (dx.east)  {};
	\draw [to] (df.south) -- (dw.north)  {};
	\draw [to] (p.west) -- (f.east) {};
	\draw [to] (w.south) -- ([xshift=0.6em]df.north) {};
	\draw [to] (x.south) -- ([xshift=-0.6em]df.north) {};
	\end{tikzpicture}
\end{center}
By construction (see eq.~\eqref{e:projected}), the function $df$ is \emph{linear} in the argument $\bp$.

Using this notation, the forward and backward passes through the original network can be rewritten as evaluating an extended network which contains a BP-reverse of the original one (in blue in the diagram):
\begin{center}
	\begin{tikzpicture}[auto, node distance=2cm]
	\node (x0) [data] {$\bx_0$};
	%
	\node (f1) [block,right of=x0] {$f_1$};
	\node (x1) [data,right of=f1] {$\bx_{1}$};
	\node (w1) [data,below of=f1] {$\bw_1$};
	%
	\node (f2) [block,right of=x1] {$f_2$};
	\node (x2) [data,right of=f2] {$\bx_{2}$};
	\node (w2) [data,below of=f2] {$\bw_2$};
	%
	\node (f3) [right of=x2] {$\dots$};
	\node (xLm) [right of=f3] {$\bx_{L-1}$};
	%
	\node (fL) [block,right of=xLm] {$f_L$};
	\node (xL) [data,right of=fL] {$\bx_{L}$};
	\node (wL) [data,below of=fL] {$\bw_L$};
	%
	\draw [to] (x0.east) -- (f1.west) {};
	%
	\draw [to] (w1.north) -- (f1.south) {};
	\draw [to] (f1.east) -- (x1.west) {};
	\draw [to] (x1.east) -- (f2.west) {};
	%
	\draw [to] (w2.north) -- (f2.south) {};
	\draw [to] (f2.east) -- (x2.west) {};
	\draw [to] (x2.east) -- (f3.west) {};
	%
	\draw [to] (f3.east) -- (xLm.west) {};
	\draw [to] (xLm.east) -- (fL.west) {};
	%
	\draw [to] (wL.north) -- (fL.south) {};
	\draw [to] (fL.east) -- (xL.west) {};
	%
	\node (dfL) [block,below of=wL,bp] {$df_L$};
	\node (dxL) [data,right of=dfL,bpe] {$d\bp_L$};
	\node (dwL) [data,below of=dfL,bpe] {$d\bw_L$};
	\node (dxLm) [data,left of=dfL,bpe] {$d\bx_{L-1}$};
	%
	\node (df3) [left of=dxLm,bpe] {$\dots$};
	%
	\node (df2) [block,below of=w2,bp] {$df_2$};
	\node (dx2) [data,right of=df2,bpe] {$d\bx_{2}$};
	\node (dw2) [data,below of=df2,bpe] {$d\bw_2$};
	%
	\node (df1) [block,below of=w1,bp] {$df_1$};
	\node (dx1) [data,right of=df1,bpe] {$d\bx_{1}$};
	\node (dw1) [data,below of=df1,bpe] {$d\bw_1$};
	%
	\node (dx0) [data,left of=df1,bpe] {$d\bx_{0}$};
	%
	\draw [to,bp] (wL.south) -- (dfL.north) {};
	\draw [to,bp] (dfL.south) -- (dwL.north) {};
	\draw [to,bp] (dxL.west) -- (dfL.east) {};
	\draw [to,bp] (dfL.west) -- (dxLm.east) {};
	%
	\draw [to,bp] (dxLm.west) -- (df3.east) {};
	\draw [to,bp] (df3.west) -- (dx2.east) {};
	%
	\draw [to,bp] (w2.south) -- (df2.north) {};
	\draw [to,bp] (df2.south) -- (dw2.north) {};
	\draw [to,bp] (dx2.west) -- (df2.east) {};
	\draw [to,bp] (df2.west) -- (dx1.east) {};
	%
	\draw [to,bp] (w1.south) -- (df1.north) {};
	\draw [to,bp] (df1.south) -- (dw1.north) {};
	\draw [to,bp] (dx1.west) -- (df1.east) {};
	%
	\draw [to,bp] (df1.west) -- (dx0.east) {};
	%
	\draw [to,bp] (x0) -- (df1) {} ;
	\draw [to,bp] (x1) -- (df2) {} ;
	\draw [to,bp] (xLm) -- (dfL) {} ;
	\end{tikzpicture}
\end{center}

% ------------------------------------------------------------------
\paragraph{Backpropagation in DAGs}\label{s:dag}
% ------------------------------------------------------------------

Assume that the DAG has a single output variable $\bx_L$ and assume, without loss of generality, that all variables are sorted in order of computation $(\bx_0,\bx_1,\dots,\bx_{L-1},\bx_L)$ according to the DAG structure. Furthermore, in order to simplify the notation, assume that this list contains both data and parameter variables, as the distinction is moot for the discussion in this section.

We can cut the DAG at any point in the sequence by fixing $\bx_0, \dots, \bx_{l-1}$ to some arbitrary value and dropping all the DAG layers that feed into them, effectively transforming the first $l$ variables into inputs. Then, the rest of the DAG defines a function $h_l$ that maps these input variables to the output $\bx_L$:
\[
\bx_L = h_l(\bx_0,\bx_1,\dots,\bx_{l-1}).
\]
Next, we show that backpropagation in a DAG iteratively computes the projected derivatives of all functions $h_1,\dots,h_L$ with respect to all their parameters.

Backpropagation starts by initializing variables $(d\bx_{0},\dots,d\bx_{l-1})$ to null tensors of the same size as $(\bx_0,\dots,\bx_{l-1})$. Next, it computes the projected derivatives of
\[
\bx_L = h_L(\bx_0,\bx_1,\dots,\bx_{L-1}) =
f_{\pi_L}(\bx_0,\bx_1,\dots,\bx_{L-1}).
\]
Here $\pi_l$ denotes the index of the layer $f_{\pi_l}$ that computes the value of the variable $\bx_l$. There is at most one such layer, or none if $\bx_l$ is an input or parameter of the original NN. In the first case, the layer may depend on any of the variables prior to $\bx_l$ in the sequence, so that general one has:
\[
\bx_{l} = f_{\pi_l}(\bx_0,\dots,\bx_{l-1}).
\]
At the beginning of backpropagation, since there are no intermediate variables between $\bx_{L-1}$ and $\bx_L$, the function $h_L$ is the same as the last layer $f_{\pi_L}$. Thus the projected derivatives of $h_L$ are the same as the projected derivatives of $f_{\pi_L}$, resulting in the equation
\[
\forall t=0,\dots,L-1:\qquad
d\bx_{t} \leftarrow d\bx_{t}
+ \frac{d\langle \bp_L, f_{\pi_L}(\bx_0,\dots,\bx_{t-1})\rangle}{d\bx_t}.
\]
Here, for uniformity with the other iterations, we use the fact that $d\bx_l$ are initialized to zero and \emph{accumulate} the values instead of storing them. In practice, the update operation needs to be carried out only for the variables $\bx_l$ that are actual inputs to $f_{\pi_L}$, which is often a tiny fraction of all the variables in the DAG.

After the update, each $d\bx_t$ contains the projected derivative of function $h_L$ with respect to the corresponding variable:
\[
\forall t=0,\dots,L-1:\qquad
d\bx_t = \frac{d\langle \bp_L, h_L(\bx_0,\dots,\bx_{l-1})\rangle}{d\bx_t}.
\]
Given this information, the next iteration of backpropagation updates the variables to contain the projected derivatives of $h_{L-1}$ instead. In general, given the derivatives of $h_{l+1}$, backpropagation computes the derivatives of $h_{l}$ by using the relation
\[
\bx_L
= 
h_{l}(\bx_0,\bx_1,\dots,\bx_{l-1})
=
h_{l+1}(\bx_0,\bx_1,\dots,\bx_{l-1},f_{\pi_L}(\bx_0,\dots,\bx_{l-1}))
\]
Applying the chain rule to this expression, for all $0\leq t \leq l-1$:
\[
\frac{d\langle \bp, h_l \rangle}{d(\vv \bx_t)^\top}
=
\frac{d\langle \bp, h_{l+1}\rangle}{d(\vv \bx_t)^\top}
+
\underbrace{\frac{d\langle \bp_L, h_{l+1}\rangle}{d(\vv \bx_l)^\top}}_{\vv d\bx_l}
\frac{d \vv f_{\pi_l}}{d(\vv \bx_t)^\top}.
\]
This yields the update equation
\begin{equation}\label{e:bp-update}	
\forall t=0,\dots,l-1:\qquad
d\bx_t \leftarrow d\bx_t + \frac{d\langle \bp_l, f_{\pi_l}(\bx_0,\dots,\bx_{l-1})\rangle}{d\bx_t},
\quad
\text{where\ }
\bp_l = d\bx_l.
\end{equation}
Once more, the update needs to be explicitly carried out only for the variables $\bx_t$ that are actual inputs of $f_{\pi_l}$. In particular, if $\bx_l$ is a data input or a parameter of the original neural network, then $\bx_l$ does not depend on any other variables or parameters and $f_{\pi_l}$ is a nullary function (i.e.\ a function with no arguments). In this case, the update does not do anything. 
After iteration $L-l+1$ completes, backpropagation remains with:
\begin{align*}
\forall t=0,\dots,l-1:&\qquad
d\bx_t
=
\frac{d\langle \bp_L, h_l(\bx_0,\dots,\bx_{l-1})\rangle}{d\bx_t}.
\end{align*}
Note that the derivatives for variables $\bx_t, l \leq t \leq L-1$ are not updated since $h_l$ does not depend on any of those. Thus, after all $L$ iterations are complete, backpropagation terminates with
\[
\forall l=1,\dots,L:\qquad
d\bx_{l-1}
=
\frac{d\langle \bp_L, h_{l}(\bx_0,\dots,\bx_{l-1})\rangle}{d\bx_{l-1}}.
\]
As seen above, functions $h_{l}$ are obtained from the original network $f$ by transforming variables $\bx_0,\dots,\bx_{l-1}$ into to inputs. If $\bx_{l-1}$ was already an input (data or parameter) of $f$, then the derivative $d\bx_{l-1}$ is applicable to $f$ as well.

Backpropagation can be summarized as follows:
\begin{center}
	\fbox{\begin{minipage}{0.95\textwidth}
			Given: a DAG neural network $f$ with a single output $\bx_L$, the values of all input variables (including the parameters), and the value of the projection $\bp_L$ (usually $\bx_L$ is a scalar and $\bp_L = p_L = 1$):
			\begin{enumerate}
				\item Sort all variables by computation order $(\bx_0,\bx_1,\dots,\bx_L)$ according to the DAG.
				\item Perform a forward pass through the network to compute all the intermediate variable values.
				\item Initialize $(d\bx_0, \dots, d\bx_{L-1})$ to null tensors with the same size as the corresponding variables.
				\item For $l=L,L-1,\dots,2,1$:
				\begin{enumerate}
					\item Find the index $\pi_l$ of the layer $\bx_{l} = f_{\pi_l}(\bx_0,\dots,\bx_{l-1})$ that evaluates variable $\bx_l$. If there is no such layer (because $\bx_{l}$ is an input or parameter of the network), go to the next iteration.
					\item Update the variables using the formula:
					\[
					\forall t=0,\dots,l-1:\qquad
					d\bx_t \leftarrow d\bx_t + \frac{d\langle d\bx_l, f_{\pi_l}(\bx_0,\dots,\bx_{l-1})\rangle}{d\bx_t}.
					\]
					To do so efficiently, use the ``backward mode'' of the layer $f_{\pi_l}$ to compute its derivative projected onto $d\bx_l$ as needed.
				\end{enumerate}
			\end{enumerate}
	\end{minipage}}
\end{center}

% TODO: what to do with multiple outputs


\begin{figure}[H]
	\begin{center}
		\begin{tikzpicture}[auto, node distance=0.3cm]
		\matrix (m) [matrix of math nodes, 
		column sep=1.2cm,
		row sep=0.3cm]
		{
			& \node (f1) [block]{f_1}; 
			& \node (x1) [datac]{\bx_1};
			\\
			\node (x0) [datac]{\bx_0};
			&
			&
			& \node (f3) [block]{f_3};
			& \node (x3) [datac]{\bx_3};
			\\
			& \node (f2) [block]{f_2}; 
			& \node (x2) [datac]{\bx_2};
			& &
			& \node (f5) [block]{f_5}; 
			& \node (x7) [datac]{\bx_7}; 
			\\
			& 
			& \node(x5) [datac]{\bx_5};
			\\
			\node (x4) [datac]{\bx_4};
			& \node (f4) [block]{f_4};
			\\
			& 
			& \node(x6) [datac]{\bx_6};
			\\
			% BP
			& \node (df1) [block,bp]{df_1}; 
			& \node (dx1) [datac,bp]{d\bx_1};
			\\
			\node (dx0) [datac,bp]{d\bx_0};
			&
			&
			& \node (df3) [block,bp]{df_3};
			& \node (dx3) [datac,bp]{d\bx_3};
			\\
			& \node (df2) [block,bp]{df_2}; 
			& \node (dx2) [datac,bp]{d\bx_2};
			& &
			& \node (df5) [block,bp]{df_5}; 
			& \node (dx7) [datac,bp]{\bp_7}; 
			\\
			& 
			& \node (dx5) [datac,bp]{d\bx_5};
			\\
			\node (dx4) [datac,bp]{d\bx_4};
			& \node (df4) [block,bp]{df_4};
			\\
			& 
			& \node(dx6) [datac,bp]{\bp_6};
			\\
		};
		\draw[to] (x0) -- (f1);
		\draw[to] (f1) -- (x1);
		\draw[to] (x1) -- (f3);
		\draw[to] (x0) -- (f2);
		\draw[to] (f2) -- (x2);
		\draw[to] (x2) -- (f3);
		\draw[to] (f3) -- (x3);
		\draw[to] (x3) -- (f5);
		\draw[to] (f5) -- (x7);
		\draw[to] (x4) -- (f4);
		\draw[to] (f4) -- (x5);
		\draw[to] (f4) -- (x6);
		\draw[to] (x5) -- (f5);
		\node(w1) [par,below=of f1]{$\bw_1$}; \draw[to] (w1) -- (f1);
		\node(w2) [par,below=of f2]{$\bw_2$}; \draw[to] (w2) -- (f2);
		\node(w4) [par,below=of f4]{$\bw_4$}; \draw[to] (w4) -- (f4);
		\draw[to] (w4) to [bend right] (f3);
		\node(w5) [par,below=of f5]{$\bw_5$}; \draw[to] (w5) -- (f5);
		\node (dx0s) [right of=dx0,xshift=20pt,draw,rectangle,bp]{$\Sigma$};
		\draw[from,bp] (dx0) -- (dx0s);
		\draw[from,bp] (dx0s) -- (df1);
		\draw[from,bp] (df1) -- (dx1);
		\draw[from,bp] (dx1) -- (df3);
		\draw[from,bp] (dx0s) -- (df2);
		\draw[from,bp] (df2) -- (dx2);
		\draw[from,bp] (dx2) -- (df3);
		\draw[from,bp] (df3) -- (dx3);
		\draw[from,bp] (dx3) -- (df5);
		\draw[from,bp] (df5) -- (dx7);
		\draw[from,bp] (dx4) -- (df4);
		\draw[from,bp] (df4) -- (dx5);
		\draw[from,bp] (df4) -- (dx6);
		\draw[from,bp] (dx5) -- (df5);
		\node(dw1) [par,below=of df1,bp]{$d\bw_1$}; \draw[from,bp] (dw1) -- (df1);
		\node(dw2) [par,below=of df2,bp]{$d\bw_2$}; \draw[from,bp] (dw2) -- (df2);
		\node(dw4s) [below of=df4,draw,rectangle,bp,yshift=-25pt]{$\Sigma$}; \draw[from,bp] (dw4s) -- (df4);
		\node(dw4) [par,below=of dw4s,bp]{$d\bw_4$}; \draw[from,bp] (dw4) -- (dw4s);
		\draw[from,bp] (dw4s) to [bend right,bp] (df3);
		\node(dw5) [par,below=of df5,bp]{$d\bw_5$}; \draw[from,bp] (dw5) -- (df5);
		%
		\draw[to,bpl] (x0) -| ([xshift=-0.3cm]x0.west) |- (df1);
		\draw[to,bpl] (x0) -| ([xshift=-0.6cm]x0.west) |- (df2);
		\draw[to,bpl] (x1) -| ([xshift=4cm]x1.west) |- ([yshift=10pt]df3.east);
		\draw[to,bpl] (x2) -| (df3);
		\draw[to,bpl] (x3) -| ([xshift=+5cm]x3.east) |- ([yshift=15pt]df5.east);
		\draw[to,bpl] (x4) to [bend right=75] ([yshift=15pt]df4.west);
		\draw[to,bpl] (x5) to [bend left] (df5);
		\end{tikzpicture}
	\end{center}
	\vspace{-1em}
	\caption{\textbf{Backpropagation network for a DAG.}}\label{f:dagbp}
\end{figure}

% ------------------------------------------------------------------
\paragraph{DAG backpropagation networks}\label{s:bpnets-dag}
% ------------------------------------------------------------------

Just like for sequences, backpropagation in DAGs can be implemented as a corresponding BP-reversed DAG. To construct the reversed DAG:
\begin{enumerate}
	\item For each layer $f_l$, and variable/parameter $\bx_t$ and $\bw_l$, create a corresponding layer $df_l$ and variable/parameter $d\bx_t$ and $d\bw_l$.
	\item If a variable $\bx_t$ (or parameter $\bw_l$) is an input of $f_l$, then it is an input of $df_l$ as well.
	\item If a variable $\bx_t$ (or parameter $\bw_l$) is an input of $f_l$, then the variable $d\bx_t$ (or the parameter $d\bw_l$) is an output $df_l$.
	\item In the previous step, if a variable $\bx_t$ (or parameter $\bw_l$) is input to two or more layers in $f$, then $d\bx_t$ would be the output of two or more layers in the reversed network, which creates a conflict. Resolve these conflicts by inserting a summation layer that adds these contributions (this corresponds to the summation in the BP update equation \eqref{e:bp-update}).
\end{enumerate}
The BP network corresponding to the DAG of \fref{f:dag} is given in \fref{f:dagbp}.


\subsection{ImageNet}\label{s:imagenet}
One of the greatest problems in AI has been in working with images and visual input. As we learned in \sref{s:cnn-structure}, convolutional neural networks are very effective at doing this. One of the great challenges that has pushed this field forward is ImageNet. ImageNet is a classification challenge where 1 million images are given and must be classified. Many groups all over the world strive to improve modern AI technology in order to top the charts annually. In order to produce more succesful results, many optimizations have been done to the standard neural network.

\subsubsection{Softmax}

The first challenge in this is how we can optimize determining which class something is. At the end of the network, we have one neuron per each output class (say 2 with dog and cat). We want to know which output class the image is. In simple cases, you can just take the highest. However, ImageNet allows you to take your top 5 best guesses since many classes turn out to be similar with thousands of classes (say different breeds of dogs). We want a way to estimate the probability of each label. To do this, we use the softmax layer, described in \sref{s:softmax}.

\subsubsection{VGG16}
\begin{figure}[H]
	\centering
	\includegraphics[scale=0.5]{images/vgg16}
	\caption{VGG16 Convolutional Network}
	\label{f:vgg16}
\end{figure}

One of the first networks that performed well combining all this so far was vgg16 \cite{pateria1990enhanced}. This network was able to classify images across thousands of categories with 90 percent accuracy given top 5 class guesses and 70.5 percent with top 1. This structure is still commonly used for many modern problems and has the benefits of being simple and easy to train. It has a slightly more powerful older cousin called vgg19 which is also very popular.

\subsubsection{ResNet}\label{s:imagenet-resnet}
\begin{figure}[H]
	\centering
	\includegraphics[width=0.65\textwidth]{images/resnetvsvgg}
	\caption{Resnet vs VGG19}
	\label{f:imagenet-resnet}
\end{figure}

For some years, the trend in machine learning was always adding more layers. With the constant improvement in GPUs, if something didn't work now, the solution was to wait a year and add more layers. However, one big issue that exists with this is the vanishing gradient. At some point, the connection between the input and the output is too spaced away and the network can no longer learn significant features at the lower layers. However, we know that deeper networks can learn more complex features more easily and are much better suited to handle hard problems. To solve this, researchers produced residual connections \cite{he2016deep}, which are described in \sref{s:blocks-resnet}.

As you can see, ResNet allows for much greater depth compared to VGG. This, too, is the smallest size of ResNet. Typical applications use ResNet-50 at least and high-end ones use ResNet-152. The sheer power of these structures has allowed for 92.9 percent accuracy on ImageNet and the usage of training on fancy GPUs with lots of data.

We indeed used ResNet-50 as backbone for our \maskrcnn.


\subsubsection{Inception}\label{s:imagenet-inception}
\begin{figure}[H]
	\centering
	\includegraphics[scale=0.5]{images/inception}
	\caption{Parallel Layers in Residual Note in Inception Neural Network.}
	\label{f:imagenet-inception}
\end{figure}

The last major improvement in recent years in image classification has been in the usage of multiple parallel layers in each residual node. Each of these sets of layers, as seen in the image above, contains different filter sizes, allowing for a varying size of features to be extracted. From this, each residual node can learn much more and increased information can be packed in. This has culminated in Inception-ResNet structures capable of producing results more accurate than humans on image classification.

Coincidentally, Tesla's neural network used for its Autonomous Driving features called “Autopilot", currently, as of V9 software (2019), still uses Inception V1.
\begin{quote}[teslamotorsclub.com]{jimmy-d}
	The V9 network takes 1280x960 images with 3 color channels and 2 frames per camera from, for example, the main camera. That’s 1280x960x3x2 as an input, or 7.3M. The V8 main camera was 640x416x2 or 0.5M - 13x less data.
\end{quote}

\subsubsection{To the Future}
In large part, following these advances, image classification has been solved. Without massive amounts of data, performance enhancements will largely be negligible. However, as these structures have gotten bigger, they have also gotten slower and slower. In recent years, the next step has been in breaking down multiple objects and analyzing complex images, as in \sref{s:nnevo}.




% ------------------------------------------------------------------
\subsection{Computational blocks}\label{s:blocks}
% ------------------------------------------------------------------

This section describes the individual computational blocks supported by \maskrcnn. The interface of a CNN computational block $<block>$ is designed after the discussion in \sref{s:fundamentals}. We can see a block as a function $y =<block>(x,w)$ that takes as input  arrays $x$ and $w$ representing the input data and parameters and returns an array $y$ as output. In general, $x$ and $y$ are 4D real arrays packing $N$ maps or images, as discussed above, whereas $w$ may have an arbitrary shape.

The function implementing each block is capable of working in the backward direction as well, in order to compute derivatives. This is done by passing a third optional argument $dzdy$ representing the derivative of the output of the network with respect to $\by$; in this case, the function returns the derivatives $[dzdx,dzdw] = <block>(x,w,dzdy)$ with respect to the input data and parameters. The arrays $dzdx$, $dzdy$ and $dzdw$ have the same dimensions of $x$, $y$ and $w$ respectively (see \sref{s:back}).

Different functions may use a slightly different syntax, as needed: many functions can take additional optional arguments, specified as property-value pairs; some do not have parameters  $w$ (e.g. a rectified linear unit); others can take multiple inputs and parameters, in which case there may be more than one $x$, $w$, $dzdx$, $dzdy$ or $dzdw$.

% ------------------------------------------------------------------
\subsubsection{Convolution}\label{s:convolution}
% ------------------------------------------------------------------

\begin{figure}[H]
	\centering
	\includegraphics[width=0.7\textwidth]{figures/svg/conv}
	\caption{\textbf{Convolution.} The figure illustrates the process of filtering a 1D signal $\bx$ by a filter $f$ to obtain a signal $\by$. The filter has $H'=4$ elements and is applied with a stride of $S_h =2$ samples. The purple areas represented padding $P_-=2$ and $P_+=3$ which is zero-filled. Filters are applied in a sliding-window manner across the input signal. The samples of $\bx$ involved in the calculation of a sample of $\by$ are shown with arrow. Note that the rightmost sample of $\bx$  is never processed by any filter application due to the sampling step. While in this case the sample is in the padded region, this can happen also without padding.}\label{f:conv}
\end{figure}

The convolutional block is implemented by the function $vl_nnconv$. $y=vl_nnconv(x,f,b)$ computes the convolution of the input map $\bx$ with a bank of $D''$ multi-dimensional filters $\bff$ and biases $b$. Here
\[
\bx\in\real^{H \times W \times D}, \quad
\bff\in\real^{H' \times W' \times D \times D''}, \quad
\by\in\real^{H'' \times W'' \times D''}.
\]
The process of convolving a signal is illustrated in \sref{f:conv} for a 1D slice. Formally, the output is given by
\[
y_{i''j''d''}
=
b_{d''}
+
\sum_{i'=1}^{H'}
\sum_{j'=1}^{W'}
\sum_{d'=1}^D
f_{i'j'd} \times x_{i''+i'-1,j''+j'-1,d',d''}.
\]
The call $vl_nnconv(x,f,[])$ does not use the biases. Note that the function works with arbitrarily sized inputs and filters (as opposed to, for example, square images). 

\subparagraph{Padding and stride.} $conv$ allows to specify  top-bottom-left-right paddings $(P_h^-,P_h^+,P_w^-,P_w^+)$ of the input array and subsampling strides $(S_h,S_w)$ of the output array:
\[
y_{i''j''d''}
=
b_{d''}
+
\sum_{i'=1}^{H'}
\sum_{j'=1}^{W'}
\sum_{d'=1}^D
f_{i'j'd} \times x_{S_h (i''-1)+i'-P_h^-, S_w(j''-1)+j' - P_w^-,d',d''}.
\]
In this expression, the array $\bx$ is implicitly extended with zeros as needed.

\subparagraph{Output size.} $conv$ computes only the ``valid'' part of the convolution; i.e. it requires each application of a filter to be fully contained in the input support.  The size of the output is given by:
\[
H'' = 1 + \left\lfloor \frac{H - H' + P_h^- + P_h^+}{S_h} \right\rfloor.
\]
Note that the padded input must be at least as large as the filters: $H +P_h^- + P_h^+ \geq H'$.

\subparagraph{Receptive field size and geometric transformations.} Very often it is useful to geometrically relate the indexes of the various array to the input data (usually images) in terms of coordinate transformations and size of the receptive field (i.e. of the image region that affects an output).

\subparagraph{Filter groups.} Filter grouping allows to group channels of the input array $\bx$ and apply different subsets of filters to each group. This takes as input a bank of $D''$ filters $\bff\in\real^{H'\times W'\times D'\times D''}$ such that $D'$ divides the number of input dimensions $D$. These are treated as $g=D/D'$ filter groups; the first group is applied to dimensions $d=1,\dots,D'$ of the input $\bx$; the second group to dimensions $d=D'+1,\dots,2D'$ and so on. Note that the output is still an array $\by\in\real^{H''\times W''\times D''}$.

An application of grouping is implementing the Krizhevsky and Hinton network (ImageNet)~\cite{krizhevsky12imagenet} which uses two such streams. Another application is sum pooling; in the latter case, one can specify $D$ groups of $D'=1$ dimensional filters identical filters of value 1 (however, this is considerably slower than calling the dedicated pooling function as given in \sref{s:pooling}).

\subparagraph{Dilation.} $conv$ allows kernels to be spatially dilated on the fly by inserting zeros between elements. For instance, a dilation factor $d=2$ causes the 1D kernel $[f_1,f_2]$ to be implicitly transformed into the kernel $[f_1,0,f_2]$. Under this notation, $d-1$ zeros are inserted between filter elements (and consequently, a dilation factor of $1$ has no effect). Thus, with dilation factors $d_h,d_w$, a filter of size $(H_f,W_f)$ is equivalent to a filter of size:
\[
H' = d_h(H_f - 1) + 1,
\qquad
W' = d_w(W_f - 1) + 1.
\]
With dilation, the convolution becomes:
\[
y_{i''j''d''}
=
b_{d''}
+
\sum_{i'=1}^{H_f}
\sum_{j'=1}^{W_f}
\sum_{d'=1}^D
f_{i'j'd} \times x_{
	S_h (i''-1)+d_h(i'-1)-P_h^-+1,
	S_w (j''-1)+d_w(j'-1)-P_w^-+1,
	d',d''}.
\]


% ------------------------------------------------------------------
\subsubsection{Convolution transpose (deconvolution)}\label{s:convt}
% ------------------------------------------------------------------

\begin{figure}[t]
	\centering
	\includegraphics[width=0.7\textwidth]{figures/svg/convt}
	\caption{\textbf{Convolution transpose.} The figure illustrates the process of filtering a 1D signal $x$ by a filter $f$ to obtain a signal $y$. The filter is applied as a sliding-window, forming a pattern which is the transpose of the one of \sref{f:conv}. The filter has $H'=4$ samples in total, although each filter application uses two of them (blue squares) in a circulant manner. The purple areas represent crops with $C_-=2$ and $C_+=3$ which are discarded. The arrows exemplify which samples of $x$ are involved in the calculation of a particular sample of $y$. Note that, differently from the forward convolution \sref{f:conv}, there is no need to add padding to the input array; instead, the convolution transpose filters can be seen as being applied with maximum input padding (more would result in zero output values), and the latter can be reduced by cropping the output instead.}\label{f:convt}
\end{figure}

The \emph{convolution transpose} block (sometimes referred to as ``deconvolution'') is the transpose of the convolution block described in \sref{s:convolution}. We'll refere to it by the name $convt$.

In order to understand convolution transpose, let:
\[
\bx\in\real^{H \times W \times D}, \quad
\bff\in\real^{H' \times W' \times D \times D''}, \quad
\by\in\real^{H'' \times W'' \times D''}, \quad
\]
be the input tensor, filters, and output tensors. Imagine operating in the reverse direction by using the filter bank $\bff$ to convolve the output $\by$ to obtain the input $\bx$, using the definitions given in~\sref{s:convolution} for the convolution operator; since convolution is linear, it can be expressed as a matrix $M$ such that  $\vv \bx = M \vv\by$; convolution transpose computes instead $\vv \by = M^\top \vv \bx$. This process is illustrated for a 1D slice in \sref{f:convt}.

There are two important applications of convolution transpose. The first one is the so called \emph{deconvolutional networks}~\cite{zeiler14visualizing} and other networks such as convolutional decoders that use the transpose of a convolution. The second one is implementing data interpolation. In fact, as the convolution block supports input padding and output downsampling, the convolution transpose block supports input upsampling and output cropping.

Convolution transpose can be expressed in closed form in the following rather unwieldy expression:
\begin{multline}\label{e:convt}
y_{i''j''d''} =
\sum_{d'=1}^{D}
\sum_{i'=0}^{q(H',S_h)}
\sum_{j'=0}^{q(W',S_w)}
f_{
	1+ S_hi' + m(i''+ P_h^-, S_h),\ %
	1+ S_wj' + m(j''+ P_w^-, S_w),\ %
	d'',
	d'
}
\times \\
x_{
	1 - i' + q(i''+P_h^-,S_h),\ %
	1 - j' + q(j''+P_w^-,S_w),\ %
	d'
}
\end{multline}
where
\[
m(k,S) = (k - 1) \bmod S,
\qquad
q(k,n) = \left\lfloor \frac{k-1}{S} \right\rfloor,
\]
$(S_h,S_w)$ are the vertical and horizontal \emph{input upsampling factors},  $(P_h^-,P_h^+,P_h^-,P_h^+)$ the \emph{output crops}, and $\bx$ and $\bff$ are zero-padded as needed in the calculation. Note also that filter $k$ is stored as a slice $\bff_{:,:,k,:}$ of the 4D tensor $\bff$.

The height of the output array $\by$ is given by
\[
H'' = S_h (H - 1) + H' -P^-_h - P^+_h.
\]
A similar formula holds true for the width. These formulas are derived in \sref{s:receptive-convolution-transpose} along with an expression for the receptive field of the operator.

We now illustrate the action of convolution transpose in an example (see also \sref{f:convt}).  Consider a 1D slice in the vertical direction, assume that the crop parameters are zero, and that $S_h>1$. Consider the output sample $y_{i''}$ where the index $i''$ is chosen such that $S_h$ divides $i''-1$; according to~\eqref{e:convt}, this sample is obtained as a weighted summation of $x_{i'' / S_h},x_{i''/S_h-1},...$ (note that the order is reversed). The weights are the filter elements $f_1$, $f_{S_h}$,$f_{2S_h},\dots$ subsampled with a step of $S_h$. Now consider computing the element $y_{i''+1}$; due to the rounding in the quotient operation $q(i'',S_h)$, this output sample is obtained as a weighted combination of the same elements of the input $x$ that were used to compute $y_{i''}$; however, the filter weights are now shifted by one place to the right: $f_2$, $f_{S_h+1}$,$f_{2S_h+1}$, $\dots$. The same is true for $i''+2, i'' + 3,\dots$ until we hit $i'' + S_h$. Here the cycle restarts after shifting $\bx$ to the right by one place. Effectively, convolution transpose works as an \emph{interpolating filter}.

\paragraph{Residual Connections}\label{s:blocks-resnet}
\begin{figure}[H]
	\centering
	\includegraphics[scale=0.5]{images/resnet}
	\caption{Single Residual Block}
	\label{f:blocks-res}
\end{figure}
Residual connections are essentially the nodes in \fref{f:blocks-res}, which consist of a set of layers. Each node extracts some features from the raw image and then combines this information with the original input to pass into the subsequent layer. In this way, features can be extracted in depth and subsequent layers can maintain access to the original information.

Residual connections are used in ResNet, described in \sref{s:imagenet-resnet}.


% ------------------------------------------------------------------
\subsubsection{Spatial pooling}\label{s:pooling}
% ------------------------------------------------------------------

Pooling can be max or avg pooling. The \emph{max pooling} operator computes the maximum response of each feature channel in a $H' \times W'$ patch
\[
y_{i''j''d} = \max_{1\leq i' \leq H', 1 \leq j' \leq W'} x_{i''+i'-1,j''+j'-1,d}.
\]
resulting in an output of size $\by\in\real^{H''\times W'' \times D}$, similar to the convolution operator of \sref{s:convolution}. Average-pooling computes the average of the values instead:
\[
y_{i''j''d} = \frac{1}{W'H'}
\sum_{1\leq i' \leq H', 1 \leq j' \leq W'} x_{i''+i'-1,j''+j'-1,d}.
\]


\subparagraph{Padding and stride.} Similar to the convolution operator of \sref{s:convolution}, $pool$ supports padding the input; however, the effect is different from padding in the convolutional block as pooling regions straddling the image boundaries are cropped. For max pooling, this is equivalent to extending the input data with $-\infty$; for sum pooling, this is similar to padding with zeros, but the normalization factor at the boundaries is smaller to account for the smaller integration area.
\pagebreak
% ------------------------------------------------------------------
\subsubsection{RoI Pooling}\label{s:roi-pooling}
% ------------------------------------------------------------------

RoI Pooling is similar to max-pooling, \sref{s:pooling}. Let's say our convolutional feature map has size $8\times8$, and our first fully connected layer needs an input of size $2\times2$. Region of Interest Pooling would then follow the figures below.

\begin{figure}[htbp]
	\centering
	\begin{minipage}{0.5\textwidth}
		\centering
		\includegraphics[width=1.1\textwidth]{images/roipool1.jpg} % first
		%figure itself
	\end{minipage}\hfill
	\begin{minipage}{0.5\textwidth}
		\centering
		\includegraphics[width=1.1\textwidth]{images/roipool2.jpg} %second
		%figure itself
	\end{minipage}
\end{figure}

\begin{figure}[htbp]
	\centering
	\begin{minipage}{0.5\textwidth}
		\centering
		\includegraphics[width=1.1\textwidth]{images/roipool3.jpg} % first
		%figure itself
	\end{minipage}\hfill
	\begin{minipage}{0.5\textwidth}
		\centering
		\includegraphics[width=0.5\textwidth]{images/roipool4.jpg} %second
		%figure itself
		\caption{$2\times2$ output.}
	\end{minipage}
\end{figure}

Simply put, if an $N\times N$ output is desired, the proposed region (black rectangle in the upper-right image) is divided into an $N\times N$ grid. When the region dimensions are not divisible by $N$, the sections of the grid will not all contain the same number of pixels. From each section, we take the greatest pixel value, and arrange these values in an $N\times N$ grid, forming our output. This output is then passed through the fully connected layers.


RoI Pooling is used in Fast \cite{Girshick_2015} and Faster R-CNN, \sref{s:nnevo-fastrcnn}
\pagebreak
\subsubsection{RoIAlign}\label{s:roi-align}
RoAlign is simply a more precise version of RoI Pooling, described above in \sref{s:roi-pooling}

\begin{figure}[H]
	\centering
	\begin{minipage}{0.5\textwidth}
		\centering
		\includegraphics[width=0.8\textwidth]{images/roialign1.PNG} % first
		%figure itself
	\end{minipage}\hfill
	\begin{minipage}{0.5\textwidth}
		\centering
		\includegraphics[width=0.8\textwidth]{images/roialign2.PNG} %second
		%figure itself
	\end{minipage}
\end{figure}

\begin{figure}[H]
	\centering
	\begin{minipage}{0.5\textwidth}
		\centering
		\includegraphics[width=0.8\textwidth]{images/roialign3.PNG} % first
		%figure itself
	\end{minipage}\hfill
	\begin{minipage}{0.5\textwidth}
		\centering
		\includegraphics[width=0.8\textwidth]{images/roialign4.PNG} %second
		%figure itself
		\caption{$2\times2$ values per cell.}
	\end{minipage}
\end{figure}

\begin{figure}[H]
	\centering
	\includegraphics[scale=0.35]{images/roialign5.PNG}
\end{figure}

Simply put, if an $N\times N$ output is desired, the proposed region (black rectangle in the upper-right image) is divided into an $N\times N$ grid. Unlike RoI Pooling, these regions will contain the exact same number of pixels, so we will often have fractional pixels. From each grid cell, we sample four regions as shown by the red $\times$ marks in the third image. We then subdivide each grid cell into four subcells, each centered on an $\times$. We perform bilinear interpolation to get a single value for each subregion, or four values for each cell. These values are shown in the fourth image. Finally, we perform a simple max pooling [\sref{s:pooling}] on the bilinear interpolated values, taking the maximum value per cell to reach an $N\times N$ output. 

You can jump back to \sref{s:maskrcnn} or understand in detail what is Bilinear Interpolation below in \sref{s:roi-bi}.

\paragraph{Bilinear Interpolation}\label{s:roi-bi}
Bilinear interpolation is fairly trivial. It is best understood visually. Simply put, the bilinearly interpolated value at the black spot is the sum of the values of each of the four colors multiplied by the areas of their respective rectangles, divided by the total area.

\begin{figure}[htbp]
	\centering
	\begin{minipage}{0.4\textwidth}
		\centering
		\includegraphics[width=0.8\textwidth]{images/bilinear.png} % first
		%figure itself
	\end{minipage}\hfill
	\begin{minipage}{0.6\textwidth}
		\centering
		\includegraphics[width=0.8\textwidth]{images/roibilinear.PNG} %second
		%figure itself
		\caption{Bilinear interpolation for RoIAlign.}
	\end{minipage}
\end{figure}

Note how in the figure on the left, the red pixel value corresponds to the smaller area opposite the pixel. This is because closer pixels (like the yellow one) have greater weighting.
The figure on the right makes it clear how bilinear interpolation is implementaed in RoIAlign. At each blue dot (represented with a red $\times$ in the figures on the previous page), we take the closest 4 pixel values and multiply them by the respective areas.

And that's all RoIAlign is. It achieves the same goal as RoI Pooling, which is to take a region of any shape and create a fixed output. However, because we are using fractional pixels, we can get much better alignment. This simple change resulted in considerable accuracy improvements for \maskrcnn, described in \sref{s:maskrcnn}

% ------------------------------------------------------------------
\subsubsection{Activation functions}\label{s:activation}
% ------------------------------------------------------------------

%
\begin{itemize}
	\item \emph{Rectified Linear Unit} (ReLU):
	\[
	y_{ijd} = \max\{0, x_{ijd}\}.
	\]
	
	\item \emph{Sigmoid.}:
	\[
	y_{ijd} = \sigma(x_{ijd}) = \frac{1}{1+e^{-x_{ijd}}}.
	\]
\end{itemize}
%



% ------------------------------------------------------------------
\subsubsection{Normalization}\label{s:normalization}
% ------------------------------------------------------------------

% ------------------------------------------------------------------
\paragraph{Local response normalization (LRN)}\label{s:ccnormalization}
% ------------------------------------------------------------------

The Local Response Normalization (LRN) operator is applied independently at each spatial location and to groups of feature channels as follows:
\[
y_{ijk} = x_{ijk} \left( \kappa + \alpha \sum_{t\in G(k)} x_{ijt}^2 \right)^{-\beta},
\]
where, for each output channel $k$, $G(k) \subset \{1, 2, \dots, D\}$ is a corresponding subset of input channels. Note that input $\bx$ and output $\by$ have the same dimensions. Note also that the operator is applied uniformly at all spatial locations.


% ------------------------------------------------------------------
\paragraph{Batch normalization}\label{s:bnorm}
% ------------------------------------------------------------------

~\cite{ioffe2015} Batch normalization is somewhat different from other neural network blocks in that it performs computation across images/feature maps in a batch (whereas most blocks process different images/feature maps individually). $y = vl_nnbnorm(x, w, b)$ normalizes each channel of the feature map $\mathbf{x}$ averaging over spatial locations and batch instances. Let $T$ be the batch size; then
\[
\mathbf{x}, \mathbf{y} \in \mathbb{R}^{H \times W \times K \times T},
\qquad\mathbf{w} \in \mathbb{R}^{K},
\qquad\mathbf{b} \in \mathbb{R}^{K}.
\]
Note that in this case the input and output arrays are explicitly treated as 4D tensors in order to work with a batch of feature maps. The tensors  $\mathbf{w}$ and $\mathbf{b}$ define component-wise multiplicative and additive constants. The output feature map is given by
\[
y_{ijkt} = w_k \frac{x_{ijkt} - \mu_{k}}{\sqrt{\sigma_k^2 + \epsilon}} + b_k,
\quad
\mu_{k} = \frac{1}{HWT}\sum_{i=1}^H \sum_{j=1}^W \sum_{t=1}^{T} x_{ijkt},
\quad
\sigma^2_{k} = \frac{1}{HWT}\sum_{i=1}^H \sum_{j=1}^W \sum_{t=1}^{T} (x_{ijkt} - \mu_{k})^2.
\]


% ------------------------------------------------------------------
\paragraph{Spatial normalization}\label{s:spnorm}
% ------------------------------------------------------------------

The spatial normalization operator acts on different feature channels independently and rescales each input feature by the energy of the features in a local neighbourhood . First, the energy of the features in a neighbourhood $W'\times H'$ is evaluated
\[
n_{i''j''d}^2 = \frac{1}{W'H'}
\sum_{1\leq i' \leq H', 1 \leq j' \leq W'} x^2_{
	i''+i'-1-\lfloor \frac{H'-1}{2}\rfloor,
	j''+j'-1-\lfloor \frac{W'-1}{2}\rfloor,
	d}.
\]
In practice, the factor $1/W'H'$ is adjusted at the boundaries to account for the fact that neighbors must be cropped. Then this is used to normalize the input:
\[
y_{i''j''d} = \frac{1}{(1 + \alpha n_{i''j''d}^2)^\beta} x_{i''j''d}.
\]


% ------------------------------------------------------------------
\paragraph{Softmax}\label{s:softmax}
% ------------------------------------------------------------------

The softmax operator is:
\[
y_{ijk} = \frac{e^{x_{ijk}}}{\sum_{t=1}^D e^{x_{ijt}}}.
\]
Note that the operator is applied across feature channels and in a convolutional manner at all spatial locations. Softmax can be seen as the combination of an activation function (exponential) and a normalization operator. 

This function essentially takes the sum of all the outputs and raises it to $e$ to make it positive. It then calculates the probability of each label by taking its value raised to $e$ over the total sum just calculated, which essentially shows us how much of the sum it consists of, nicely giving us a probability.


% ------------------------------------------------------------------
\subsubsection{Categorical losses}\label{s:losses}
% ------------------------------------------------------------------

The purpose of a categorical loss function $\ell(\bx,\bc)$ is to compare a prediction $\bx$ to a ground truth class label $\bc$. The contribution of different samples are summed together (possibly after weighting) and the output of the loss is a scalar. \sref{s:loss-classification} losses useful for multi-class classification and the \sref{s:loss-attributes} losses useful for binary attribute prediction. 

% ------------------------------------------------------------------
\paragraph{Classification losses}\label{s:loss-classification}
% ------------------------------------------------------------------

Classification losses decompose additively as follows:
\begin{equation}\label{e:addloss}
\ell(\bx,\bc) = \sum_{ijn} w_{ij1n} \ell(\bx_{ij:n}, \bc_{ij:n}).
\end{equation}
Here $\bx \in \mathbb{R}^{H \times W \times C \times N}$ and $\bc \in \{1, \dots, C\}^{H \times W \times 1 \times N}$, such that the slice $\bx_{ij:n}$ represent a vector of $C$ class scores and and $c_{ij1n}$ is the ground truth class label. The $`instanceWeights`$ option can be used to specify the tensor $\bw$ of weights, which are otherwise set to all ones; $\bw$ has the same dimension as $\bc$.

\subparagraph{Classification error.} The classification error is zero if class $c$ is assigned the largest score and zero otherwise:
\begin{equation}\label{e:loss-classerror}
\ell(\bx,c) = \mathbf{1}\left[c \not= \argmax_k x_c\right].
\end{equation}
Ties are broken randomly.

\subparagraph{Top-$K$ classification error.} The top-$K$ classification error is zero if class $c$ is within the top $K$ ranked scores:
\begin{equation}\label{e:loss-classerror}
\ell(\bx,c) = \mathbf{1}\left[ |\{k : x_k \geq x_c \}| \leq K \right].
\end{equation}
The classification error is the same as the top-$1$ classification error.

\subparagraph{Log loss or negative posterior log-probability.} In this case, $\bx$ is interpreted as a vector of posterior probabilities $p(k) = x_k, k=1,\dots, C$ over the $C$ classes. The loss is the negative log-probability of the ground truth class:
\begin{equation}\label{e:loss-log}
\ell(\bx, c) = - \log x_c.
\end{equation}
Note that this makes the implicit assumption $\bx \geq 0, \sum_k x_k = 1$. Note also that, unless $x_c > 0$, the loss is undefined. For these reasons, $\bx$ is usually the output of a block such as softmax that can guarantee these conditions. However, the composition of the naive log loss and softmax is numerically unstable. Thus this is implemented as a special case below.

Generally, for such a loss to make sense, the score $x_c$ should be somehow in competition with the other scores $x_k, k\not = c$. If this is not the case, minimizing \eqref{e:loss-log} can trivially be achieved by maxing all $x_k$ large, whereas the intended effect is that $x_c$ should be large compared to the $x_k, k\not=c$. The softmax block makes the score compete through the normalization factor.

\subparagraph{Softmax log-loss or multinomial logistic loss.} This loss combines the softmax block and the log-loss block into a single block:
\begin{equation}\label{e:loss-softmaxlog}
\ell(\bx, c) = - \log \frac{e^{x_c}}{\sum_{k=1}^C e^{x_k}}
= - x_c + \log \sum_{k=1}^C e^{x_k}.
\end{equation}
Combining the two blocks explicitly is required for numerical stability. Note that, by combining the log-loss with softmax, this loss automatically makes the score compete: $\ell(bx,c) \approx 0$ when $x_c \gg \sum_{k\not= c} x_k$.

\subparagraph{Multi-class hinge loss.} The multi-class logistic loss is given by
\begin{equation}\label{e:loss-multiclasshinge}
\ell(\bx, c) = \max\{0, 1 - x_c \}.
\end{equation}
Note that $\ell(\bx,c) =0 \Leftrightarrow x_c \geq 1$. This, just as for the log-loss above, this loss does not automatically make the score competes. In order to do that, the loss is usually preceded by the block:
\[
y_c = x_c - \max_{k \not= c} x_k.
\]
Hence $y_c$ represent the \emph{confidence margin} between class $c$ and the other classes $k \not= c$. Just like softmax log-loss combines softmax and loss, the next loss combines margin computation and hinge loss.

\subparagraph{Structured multi-class hinge loss.} The structured multi-class logistic loss, also know as Crammer-Singer loss, combines the multi-class hinge loss with a block computing the score margin:
\begin{equation}\label{e:loss-structuredmulticlasshinge}
\ell(\bx, c) = \max\left\{0, 1 - x_c + \max_{k \not= c} x_k\right\}.
\end{equation}

% ------------------------------------------------------------------
\paragraph{Attribute losses}\label{s:loss-attributes}
% ------------------------------------------------------------------

Attribute losses are similar to classification losses, but in this case classes are not mutually exclusive; they are, instead, binary attributes. Attribute losses decompose additively as follows:
\begin{equation}\label{e:addlossattribute}
\ell(\bx,\bc) = \sum_{ijkn} w_{ijkn} \ell(\bx_{ijkn}, \bc_{ijkn}).
\end{equation}
Here $\bx\in \mathbb{R}^{H \times W \times C \times N}$ and $\bc \in \{-1,+1\}^{H \times W \times C \times N}$, such that the scalar $x_{ijkn}$ represent a confidence that attribute $k$ is on and $c_{ij1n}$ is the ground truth attribute label. The $`instanceWeights`$ option can be used to specify the tensor $\bw$ of weights, which are otherwise set to all ones; $\bw$ has the same dimension as $\bc$.

Unless otherwise noted, we drop the other indices and denote by $x$ and $c$  the scalars $x_{ijkn}$ and  $c_{ijkn}$. As before, samples with $c=0$ are skipped.

\subparagraph{Binary error.} This loss is zero only if the sign of $x - \tau$ agrees with the ground truth label $c$:
\begin{equation}\label{e:loss-binary}
\ell(x,c|\tau) = \mathbf{1}[\sign(x-\tau) \not= c].
\end{equation}
Here $\tau$ is a configurable threshold, often set to zero.

\subparagraph{Binary log-loss.} This is the same as the multi-class log-loss but for binary attributes. Namely, this time $x_k \in [0,1]$ is interpreted as the probability that attribute $k$ is on:
\begin{align}\label{e:loss-binarylogloss}
\ell(x,c)
&=
\begin{cases}
- \log x, & c = +1, \\
- \log (1 - x), & c = -1, \\
\end{cases}
\\
&=
- \log \left[ c \left(x - \frac{1}{2}\right) + \frac{1}{2} \right].
\end{align}
Similarly to the multi-class log loss, the assumption $x \in [0,1]$ must be enforced by the block computing $x$.

\subparagraph{Binary logistic loss.} This is the same as the multi-class logistic loss, but this time $x/2$ represents the confidence that the attribute is on and $-x/2$ that it is off. This is obtained by using the logistic function $\sigma(x)$
\begin{equation}\label{e:loss-binarylogistic}
\ell(x,c)
=
- \log \sigma(cx)
=
-\log \frac{1}{1 + e^{-{cx}}}
=
-\log \frac{e^{\frac{cx}{2}}}{e^{\frac{cx}{2}} + e^{-\frac{cx}{2}}}.
\end{equation}

\subparagraph{Binary hinge loss.} This is the same as the structured multi-class hinge loss but for binary attributes:
\begin{equation}\label{e:loss-hinge}
\ell(x,c)
=
\max\{0, 1 - cx\}.
\end{equation}
There is a relationship between the hinge loss and the structured multi-class hinge loss which is analogous to the relationship between binary logistic loss and multi-class logistic loss. Namely, the hinge loss can be rewritten as:
\[
\ell(x,c) = \max\left\{0, 1 - \frac{cx}{2} + \max_{k\not= c} \frac{kx}{2}\right\}
\]
Hence the hinge loss is the same as the structure multi-class hinge loss for $C=2$ classes, where $x/2$ is the score associated to class $c=1$ and $-x/2$ the score associated to class $c=-1$.

% ------------------------------------------------------------------
\section{Pattern Recognition Methods}\label{s:patt}
% ------------------------------------------------------------------

Following from \sref{s:ltask}, in which we talked about all the problems NNs solve for us, we'll describe in this section \emph{Pattern Recognition} speficically, which was introduced in \sref{s:ltask-patt}.


\subsection{Image Classification}\label{s:patt-class}
Image classification is the process of assigning land cover classes to pixels. Image classification refers to the task of extracting information classes from a multiband raster image. The resulting raster from image classification can be used to create thematic maps. Depending on the interaction between the analyst and the computer during classification, there are two types of classification: supervised and unsupervised. The image classification plays an important role in environmental and socioeconomic applications. In order to improve the classification accuracy, scientists have laid path in developing the advanced classification techniques.
Image classification analyzes the numerical properties of various image features and organizes data into categories. Classification algorithms typically employ two phases of processing: training and testing. In the initial training phase, characteristic properties of typical image features are isolated and, based on these, a unique description of each classification category, i.e. training class, is created. In the subsequent testing phase, these feature-space partitions are used to classify image features. The description of training classes is an extremely important component of the classification process. In supervised classification, statistical processes (i.e. based on an a priori knowledge of probability distribution functions) or distribution-free processes can be used to extract class descriptors. Unsupervised classification relies on clustering algorithms to automatically segment the training data into prototype classes. In either case, the motivating criteria for constructing training classes are that they are:
\begin{enumerate}
	\item Independent, e.a change in the description of one training class should not change the value of another,
	\item Discriminatory, e.different image features should have significantly different descriptions, and
	\item Reliable, all image features within a training group should share the common definitive descriptions of that group.
\end{enumerate}

A convenient way of building a parametric description of this sort is via a feature vector $v1,v2,\dots,vn$ where $n$ is the number of attributes which describe each image feature and training class. This representation allows us to consider each image feature as occupying a point, and each training class as occupying a sub-space (i.e. a specific class amongst all the classes), within the n-dimensional classification space. Viewed as such, the classification problem is that of determining to which sub-space class each feature vector (i.e. all the pixels) belongs.
\begin{figure}[H]
	\centering
	\includegraphics[width=0.7\linewidth]{images/classification_detection_segmentaion_comparisons.jpeg}
	\caption{Image Classification,Object detection Semantic Segmentation.}
\end{figure}
\subsection{Object Detection}\label{s:patt-dtct}
The goal of object detection is to detect all instances of objects from a known
class, such as people, cars or faces in an image. Typically only a small number
of instances of the object are present in the image, but there is a very large
number of possible locations and scales at which they can occur and that need
to somehow be explored \cite{chavan2016object}.
Each detection is reported with some form of pose information. This could
be as simple as the location of the object, a location and scale, or the extent
of the object defined in terms of a bounding box. In other situations the pose
information is more detailed and contains the parameters of a linear or non-linear
transformation. For example a face detector may compute the locations of the
eyes, nose and mouth, in addition to the bounding box of the face. An example
of a vehicle and person detection that specifies the locations of certain parts is shown in
\fref{f:patt-dtct}. The pose could also be defined by a three-dimensional transformation
specifying the location of the object relative to the camera.
Object detection systems construct a model for an object class from a set of
training examples. In the case of a fixed rigid object only one example may be
needed, but more generally multiple training examples are necessary to capture
certain aspects of class variability.
\begin{figure}[H]
	\centering
	\includegraphics[width=0.5\linewidth]{images/object_det.jpeg}
	\caption{Object detection with bounding boxes.}
	\label{f:patt-dtct}
\end{figure}

Object detection methods fall into two major categories, generative (we use GAN\footnote{Generative Adversarial Network}) and discriminative (we use CNN). The first consists of a probability model for the image appearance on a given pose, together with a model for background, i.e. non-object images. The model parameters can be estimated from training data and the decisions are based on ratios of posterior probabilities \footnote{A posterior probability is the probability of assigning observations to groups given the data. A prior probability is the probability that an observation will fall into a group before you collect the data. For example, if you are classifying the buyers of a specific car, you might already know that 60\% of purchasers are male and 40\% are female. If you know or can estimate these probabilities, a discriminant analysis can use these prior probabilities in calculating the posterior probabilities. So a posterior probability uses known data to predict the probability the object will fall in that specific class}.
The second typically builds a classifier that can discriminate between images (or sub-images) containing the object and those not containing the object. The parameters of the classifier are selected to minimize mistakes on the training data, often with a regularization bias to avoid overfitting [\fref{f:overfitting}]. Other distinctions among detection algorithms have to do with the computational tools used to scan the entire image or search over possible poses, the type of image representation with which the models are constructed, and what type and how much training data is required to build a model.	

All the nets that preceded \maskrcnn only did Object Detection, as described in \sref{s:nnevo}.




\subsection{Semantic Segmentation}\label{s:patt-sema}
Segmentation is essential for image analysis tasks. Semantic segmentation describes the process of associating each pixel of an image with a class label, (such as flower, person, road, sky, ocean, or car).
Semantic image segmentation can be applied effectively to any task that involves the segmentation of visual information. Examples include road segmentation for autonomous vehicles, medical image segmentation, scene segmentation for robot perception, and in image editing tools. Whilst currently available systems provide accurate object recognition, they are unable to delineate the boundaries between objects with the same accuracy.

Oxford researchers have developed a novel neural network component for semantic segmentation that enhances the ability to recognise and delineate objects. This invention can be applied to improve any situation requiring the segmentation of visual information.

Semantic image segmentation plays a crucial role in image understanding, allowing a computer to recognise objects in images. Recognition and delineation of objects is achieved through classification of each pixel in an image. Such processes have a wide range of applications in computer vision, in diverse and growing fields such as vehicle autonomy and medical imaging.

The previous state-of-the-art image segmentation systems used Fully Convolutional Neural Network (FCNN) components, which offer excellent accuracy in recognising objects. Whilst this development represented a significant improvement in semantic segmentation, these networks do not perform well in delineating object boundaries.
\begin{figure}[H]
	\centering
	\includegraphics[width=0.5\linewidth]{images/semantic.jpg}
	\caption{Image with semantic segmentation.}
\end{figure}
Examples include, but are not limited to, road segmentation for autonomous vehicles, medical image segmentation, scene segmentation for robot perception and image editing tools.


\subsection{Instance Segmentation}\label{s:patt-insta}
Instance segmentation is one step ahead of semantic segmentation wherein along with pixel level classification, we expect the computer to classify each instance of a class separately. For example in the image above there are 3 people, technically 3 instances of the class “Person”. All the 3 are classified separately (in a different color). But semantic segmentation does not differentiate between the instances of a particular class.
\begin{figure}[H]
	\centering
	\includegraphics[width=0.5\linewidth]{images/instance.png}
	\caption{Image with instance segmentation.}
\end{figure}

That is what \maskrcnn does and is the best at, and why we chose it to pursue apparel recognition in real world images.



% ------------------------------------------------------------------
\section{NN evolution for Object Detection}\label{s:nnevo}
% ------------------------------------------------------------------

\subsection{Introduction}\label{s:nnevo-intro}

Images and videos are collected everyday by different sources. Recognizing objects, segmenting localizing and classifying them has been a major area of interest in computer vision. 

After having explained the basic features of Neural Networks [\sref{s:perc}] and CNNs [\sref{s:cnn-structure}] and having understood the task we're confronting with [Object Detection, \sref{s:patt-dtct}], we are ready to dive into how the algorithms for solving this problem have evolved in the last decades.

\subsection{Convolutional Neural Network}\label{s:nnevo-cnn}

Just as Convolutional Neural Network (CNN) is traced to the Fukushima’s \emph{ “neocognitron”} \cite{fukushima1980neocognitron}, a hierarchical and shift-invariant model for pattern recognition, the use of CNN for region-based identification (R-CNN)\cite{Girshick_2014} can also be traced back to the same.  After CNN was considered inelegant in the 1990s due to the rise of support vector machine (SVM, \fref{fig:svm}), in 2012 it was revitalize by Krizhevsky et al. \cite{krizhevsky2012imagenet} by demonstrating a valuable improvement in image classification accuracy on the ImageNet Large Scale Visual Recognition Challenge (ILSVRC), explained in \ref{s:imagenet}, and included new mechanisms to CNN like rectified linear unit (ReLU) and, dropout regularization. 

\subsection{Region-based CNN}\label{s:nnevo-rcnn}

To perform object detection with CNN and in attempt to bridge the gap between image segmentation and object detection two issues were fixed by R.Girshick et al \cite{Girshick_2014}. First was the localization of objects with a Deep Network and training a high-capacity model with only a small quantity of annotated detection data. Use of a sliding-window detector was proposed for the localization of object but was not preferred because it can only work for one object detection and all objects in an image have to have a common aspect ratio for its use in multiple object detection. Instead the localization problem was solved by operating within the “recognition using regions” paradigm.

The Region-based Convolutional Neural Network (R-CNN) is relatively simple. First, we propose some regions of an image which contain objects. We feed these regions into a CNN, and then we use an SVM to classify the outputs for each class.

\begin{center}
	\includegraphics[scale=0.3]{images/rcnn.png}
\end{center}

If we wish to have a CNN classify objects in images, we need to feed in a region of the image to the CNN. Of course, the question becomes: How do we know which regions to feed into a network? We cannot possibly try every single possible region of an image; there are too many combinations. We must have a way to propose regions which are likely to contain objects.

\begin{figure}[H]
	\centering
	\begin{minipage}{0.45\textwidth}
		\centering
		\includegraphics[width=0.66\textwidth]{images/search.PNG} % first
		\caption{Too many regions!}
		\label{f:regionscnn}
		%figure itself
	\end{minipage}\hfill
	\begin{minipage}{0.45\textwidth}
		\centering
		\includegraphics[width=0.66\textwidth]{images/region.PNG} % second figure itself
	\end{minipage}
\end{figure}

R-CNN is agnostic to the region proposal algorithm. The original R-CNN uses the Selective Search algorithm. Since selective search, various region proposal methods have been developed.

Selective Search detects image segments to propose regions. It uses the segments it finds to identify the containing boxes, which will be used by R-CNN as regions.

\begin{figure}[H]
	\centering
	\includegraphics[scale=0.4]{images/segment.PNG}
	\caption{Using segments to define regions}
	\label{f:segment}
\end{figure}

\subsection{Fast R-CNN}\label{s:nnevo-fastrcnn}

Fast R-CNN was introduced in 2015 by Girshick \cite{Girshick_2015}. A single-stage training algorithm that jointly learns to classify object proposals and refine their spatial locations was demonstrated.

\begin{figure}[H]
	\centering
	\includegraphics[scale=0.35]{images/fastrcnn.PNG}
	\caption{Fast R-CNN Architecture}
	\label{f:fastrcnn}
\end{figure}

Fast R-CNN, instead of running the CNN for each region, runs it just a single time. And then takes the regions from the convolutional feature map (the output of a filter applied to the previous layer).

This tackled the problem of complexity that arises in other deep ConvNets, caused by the multi-stage pipelines that are slow. The slow nature is due to the fact that detection requires accurate localization of objects that creates the challenge of that many proposals (candidate object locations) must be processed and these proposals provides only rough localization that must be refined to achieve precise localization. Fast R-CNN is 9 X faster than R-CNN \cite{Girshick_2014} and 3 X faster than SPPnet \cite{he2015spatial}. R-CNN was sped up by \emph{Spatial pyramid pooling networks (SPPnets)}\cite{he2015spatial} by sharing computation. A convolutional feature map for the entire input image was computed by SPPnet method. After which it then classifies each object proposal using a feature vector extracted from the shared feature map. 

\fref{f:fast-rcnn-arch} below makes it clear that Fast R-CNN proposes regions of interest ``RoI" from the input image. These are generated using a region proposal algorithm (e.g. Selective Search), just like they were in R-CNN.

\begin{figure}[H]
	\centering
	\includegraphics[scale=0.55]{images/fast-rcnn-arch.png}
	\caption{Fast R-CNN RoI Projection}
	\label{f:fast-rcnn-arch}
\end{figure}

However, in Fast R-CNN, as \fref{f:fastrcnn} shows, these regions of interest are then projected onto the convolutional feature map. This allows us to re-use the expensive convolutional computation. We take crops from this feature map and run them through the rest of the network. The question then becomes: how exactly do we \textit{project} a region of the input image onto a region of the convolutional feature map?

The SPPNet paper (``Spatial Pyramid Pooling in Deep Convolutional
Networks for Visual Recognition" by He et al.)~\cite{he2015spatial}, which came out between R-CNN and Fast R-CNN, first introduced the idea of RoI projection.

RoI Projections are of course different from cropping an image, so here comes RoI Pooling, which is described in \sref{s:roi-pooling}.

SPPnet also has obvious pitfalls. It is a multi-stage pipeline similar to R-CNN that involves extracting features, refining a network with log loss, training SVMs, and lastly fitting bounding-box regressors. Features are also written to disk. But unlike R-CNN, the refining algorithm demonstrated in SPPnet cannot update the convolutional layers that precede the spatial pyramid pooling. This constraint limits the accuracy of very deep networks.

Even though SVMs worked better than Softmax [\sref{s:softmax}] in R-CNN, the opposite is true for Fast R-CNN. This was demonstrated only empirically, as per \tref{t:fastrcnn-svmvssoftmax}

\begin{table}[H]
	\centering
	\begin{tabular}{c c c c} 
		\hline
		Method (VGG16) & Classifier & VOC07 \\ [0.5ex] 
		\hline
		R-CNN & Post-hoc SVM & 66.0\% \\
		Fast R-CNN & Post-hoc SVM & 66.8\% \\
		Fast R-CNN & Softmax & 66.9\% \\
		\hline
	\end{tabular}
	\caption{SVM vs Softmax for training the probability distribution for each RoI}
	\label{t:fastrcnn-svmvssoftmax}
\end{table}

Also, Fast R-CNN greatly improves upon the speed of R-CNN -- so much so that the region proposal algorithm takes the vast majority of time during testing, as shown below in \fref{f:chart}.

\begin{figure}[H]
	\centering
	\includegraphics[width=0.75\textwidth]{images/chart.png} %second
	\caption{Training Time R-CNN, SPPNet and Fast R-CNN. Blue bars include region proposal algorithm.}
	\label{f:chart}
\end{figure}

Faster R-CNN sets out to solve the problem of slow region proposal!

\subsection{Faster R-CNN}\label{s:nnevo-fasterrcnn}

Additional efforts were made to reduce the running time of deep ConvNets for object detection and segmentation. Regional proposal computation is the root of this expensive running time in detection networks. Even the fast EdgeBoxes algorithm takes 0.3 seconds, as in \fref{f:table} below.

\begin{figure}[H]
	\centering
	\includegraphics[scale=0.5]{images/table.PNG}
	\caption{Region Proposal Algorithms comparison.}
	\label{f:table}
\end{figure}

A fully convolutional network that simultaneously predicts object bounds and objectness scores at each position called \emph{ Region Proposal Network (RPN) }was developed by Ren et al \cite{ren2015faster}. RPN shares full-image convolutional features with the detection network, thus permitting virtually cost-free region proposals and it is trained end-to-end to generate high-quality region proposals.

To put it simply, it leverages the filter map output we used in Fast R-CNN to propose regions. This time, though, the proposing algorithm is a neural net that learns to predict the more plausible regions, and not a fixed algorithm such as the ones above in \fref{f:chart}.
After all, if convolutions are good enough for classification and bounding-box regression, why wouldn't they work for region proposals as well?

Integrating RPN and Fast R-CNN into a unit network by sharing their convolutional features results to Faster R-CNN.

% TODO RPN in detail

Anchor boxes that acts as reference at multiple scales and aspect ratios were introduced in Faster R-CNN instead of the pyramids of filters used in earlier methods. RPNs are developed to coherently speculate region proposals with an extensive range of scales and aspect ratios.

Changing the architecture of the pyramids of filter to a top-down architecture with lateral connections improved the efficiency of this pyramids \cite{lin2017feature}. This is applied in building high-level semantic feature maps at all scales. This new architecture is called \emph{ Feature Pyramid Network (FPN)} \cite{lin2017feature}, shown in comparison with ResNet in \sref{s:maskrcnn-maskhead}. In various applications and uses it displayed a notable improvement as a generic feature extractor. When used in a Faster R-CNN it achieved results that supersedes that of Faster R-CNN alone. 

\subsection{Mask R-CNN}\label{s:nnevo-maskrcnn}

In order to generate a high-quality segmentation mask for object instances in an image, \maskrcnn was developed \cite{He_2017}. \maskrcnn adds another branch to the Faster R-CNN. In addition to the bounding box recognition system a branch for predicting an object mask in parallel was added. It affixes only a bijou overhead to Faster R-CNN. 

This allows for our task: Instance Segmentation, \sref{s:maskrcnn}, in which I'll describe in more detail the features of \maskrcnn.


\section{Instance Segmentation}\label{s:maskrcnn}

Instance Segmentation is one of the most difficult image-based computer vision tasks. It combines elements of semantic segmentation (pixel-level classification) and object detection (instance recognition). Essentially, at every pixel, we wish to classify not only the type of object (or background) the pixel is part of, but also determine which instance the pixel is part of, as described in \sref{s:patt}

\begin{figure}[H]
	\centering
	\includegraphics[scale=0.3]{images/tasks.PNG}
	\caption{Pattern recognition tasks.\newline Source: Fei-Fei Li Stanford Course -- Detection And Segmentation}
	\label{f:tasks}
\end{figure}

Not unsurprisingly, instance segmentation networks rely heavily on existing object detection networks. \maskrcnn modifies the Faster R-CNN architecture and adapts it for instance segmentation with minimal overhead. \maskrcnn is the current leader in instance segmentation performance.

A previous ``fully convolutional instance segmentation" (FCIS) solution, which performed segmentation, classification, and bounding-box regression simultaneously, although very fast, exhibited low segmentation accuracy, especially on overlapping objects. \maskrcnn therefore takes a different approach, \textit{decoupling} segmentation from classification and bounding-box regression. \maskrcnn thus adds a separate mask ``head" to the Faster R-CNN network. This is shown in the diagram below.

The mask ``head" is simply a small fully convolutional network that outputs an $m\times m$ mask for each region proposal. We use a fully convolutional network rather than fully connected layers so we do not lose spatial information. A fully convolutional solution requires fewer parameters than previous fully connected solutions while simultaneously increasing accuracy.

The two diagrams at the bottom of \fref{f:fastermaskrcnn} show a different visualization of Faster R-CNN and \maskrcnn. Besides the additional ``mask" head of \maskrcnn, we can see that RoIAlign replaces RoI Pooling and is described in \sref{s:roi-align}

\begin{figure}[H]
	\centering
	\includegraphics[scale=0.36]{images/masknetwork.png}

	\includegraphics[scale=0.45]{images/fasterrcnnvs.PNG}

	\includegraphics[scale=0.45]{images/maskrcnn.PNG}

	\caption{Faster R-CNN vs \maskrcnn}
	\label{f:fastermaskrcnn}
\end{figure}

The output of RoI Align then passed through the fully connected layers for bounding-box regression and classification, and through the small Fully Convolutional Network (FCN) that makes up the masking head.

\subsection{Mask Head}\label{s:maskrcnn-maskhead}
Depending on the network backbone, the mask head differs for \maskrcnn. Below is a look at the two different heads. Both are trivial FCNs.

\begin{figure}[H]
	\centering
	\includegraphics[scale=0.3]{images/backbones.PNG}
	\caption{Comparison between Faster R-CNN with ResNet or FPN backbone}
\end{figure}

In the diagram above, FPN stands for ``Feature Pyramid Network", introduced in \sref{s:nnevo-fasterrcnn}. ResNet, meanwhile, is described in \sref{s:imagenet-resnet}.

There are a few important things to know about this mask head. First, like we said earlier, our output is an $m\times m$ mask. However, the authors found it beneficial to have binary masks. In other words, we predict $K$ $m\times m$ masks for each RoI, where $K$ is the number of classes. One mask per class. Thus, the mask branch has a $Km^2$-dimensional output for each region of interest.

Our loss function is now different. Previously, we had $L = L_{boundingbox} + L_{classification}$. We've covered the details of classification and bounding-box loss in our object detection, for both the region proposal network (RPN) and the network itself. For \maskrcnn, we add another loss, $L_{mask}$. For some region $r$, if the ground truth class is $k$, we apply a per-pixel sigmoid on \textit{only} the $k$th mask. This allows us to define $L_{mask}$ as the average binary cross-entropy loss. Thus, the masks for classes that don't correspond to the ground truth aren't calculated. (remember, we have one mask per class for every region).

By computing one mask per class, we are \textit{decoupling} classification and segmentation. We simply don't care what class the object is when we segment it. Previous practices, like FCNs for semantic segmentation, use multi-class cross-entropy losses and per-pixel softmax. These allows for competition between classes, which \maskrcnn eliminates.

\subsection{Training and Testing}
When training, \maskrcnn shares similarities with its object detection cousins. Hyperparameters were set to the same values. Positive RoIs have IoU, described in \fref{f:iou}, of at least 0.5 with the ground truth box. In addition, $L_{mask}$ is defined only on positive RoIs. The mask target is the intersection between an RoI and its associated ground-truth mask. 

At test time, after non-maximum supression is applied, the masking branch is applied on only the top 100 RoIs. If an region was classified into class $k$, we simply choose the $k$th mask. The mask is then resized to the size of the region of interest. By reducing segmentation computation to only 100 regions, we dramatically decrease the amount of overhead. In fact, \maskrcnn runs at 5 fps, compared to Faster R-CNN's 7 fps.

\subsection{Model Performance}
The \maskrcnn paper \cite{He_2017} not only provides evidence that their model outperforms all previous models, but also conducted various ablation experiments to show that RoIAlign [\sref{s:roi-align}], segmentation decoupling, and fully convolutional mask heads [\sref{s:maskrcnn-maskhead}] each individually improve accuracy. The results are shown in the tables below.


\begin{figure}[htbp]
	\centering
	\begin{minipage}{0.55\textwidth}
		\centering
		\includegraphics[width=1\textwidth]{images/moreroialigncomparison.PNG} % first
		%figure itself
	\end{minipage}\hfill
	\begin{minipage}{0.35\textwidth}
		\centering
		\includegraphics[width=1\textwidth]{images/binarymaskcomparison.PNG} %second
		%figure itself
	\end{minipage}
\end{figure}

\begin{center}
	\includegraphics[scale=0.4]{images/fcncomparison.PNG}
\end{center}

In addition, \maskrcnn performs better with a deeper backbone CNN. However, it should be noted that the 5 fps speed was achieved using the shallow ResNet-50 network as a backbone. (If you really want to call it “shallow")

\begin{center}
	\includegraphics[scale=0.4]{images/backbonecomparison.PNG}
\end{center}

The results on the COCO and Cityscapes benchmarks are shown below. \maskrcnn performs with state-of-the-art accuracy on both.
\begin{figure}[htbp]
	\centering
	\includegraphics[width=0.85\textwidth]{images/cocoresults.PNG} % first
	\caption{COCO results.}
\end{figure}
\begin{figure}[htbp]
	\centering
	\includegraphics[width=1\textwidth]{images/cityscapesresults.PNG} % first
	\caption{Cityscapes results.}
\end{figure}


\begin{figure}[H]
	\centering
	\begin{minipage}{0.35\textwidth}
		\centering
		\includegraphics[width=1\textwidth]{images/cocoexamples.PNG} % first
		%figure itself
	\end{minipage}\hfill
	\begin{minipage}{0.65\textwidth}
		\centering
		\includegraphics[width=1\textwidth]{images/cityscapes.PNG} %second
		%figure itself
	\end{minipage}
	\caption{Segmentation examples from COCO (left) and Cityscapes (right)}
	\label{f:maskrcnn-instancecity}
\end{figure}
\maskrcnn can also be used for human pose estimation.

\subsection{Conclusion}
\maskrcnn leaps ahead of the competition in terms of pure instance segmentation performance. However, no current instance segmentation method can achieve great results while operating in real-time (60 FPS). In the future, look for networks which dramatically improve segmentation speed as well as accuracy.



% ------------------------------------------------------------------
