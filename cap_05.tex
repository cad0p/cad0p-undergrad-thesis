\chapter{Conclusion and future developements}\label{s:conclusion}

This experience has been what I hoped it would be: it opened new paths and fullfilled my curiosity to learn how deep learning works and how you can use it to do many awesome things.

Putting the foundations to a new project is never an easy task, and I feel honoured to have had this opportunity. Now that the dataset is ready to be expanded and used for data analysis purposes on social media, my biggest joy would be to know that someone will work on this, so that this work has not been done for nothing, but it has paved the way for new, maybe even more exciting than I thought here in \sref{s:ds-paperdoll-raw}.

The opportunities are endless: mixing the dataset of studio quality images of shoes models, with the understanding of what a pair of shoes is -- thanks to this work -- allows for tracking how this specific model of shoes is performing on social media, and even track how Influencers/Creators convert their publicity to sales (based on how many fans post photos of that same pair of shoes).

This very same algotithm, \maskrcnn, has been used to track house roofs' exposure to the sun to better estimate costs for a solar panel installation, improving maps, allowing surgery robots to operate autonomously, and many more.

The \modanet dataset has just been released in its  entirety to the public (it's only been a few months), so I expect there will be for sure someone better than me that will fix the annotations once and for all, but in the meantime, I think the improvements are there and they make the difference between “the program is not able to detect footwear and boots" and “the program usually detects most of footwear and boots".

